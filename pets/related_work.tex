\section{Related Work}
\label{sec:related}

To our knowledge, there has been no prior work on preventing surveillance at CDNs, but there 
has been relevant research on securing CDNs and finding security vulnerabilities in CDNs.\\
\vspace{0pt}
\textbf{Securing CDNs.} Most prior work on securing CDNs has focused on providing content 
integrity at the CDN as opposed to content confidentiality (and unlinkability).  In 2005, 
Lesniewski-Laas and Kaashoek use SSL-splitting --- a technique 
where the proxy simulates an SSL connection with the client by using authentication records from 
the server with data records from the cache (in the proxy) --- to maintain the 
integrity of content being served by a proxy~\cite{lesniewski2005ssl}.  While this work does not 
explicitly apply SSL-splitting to CDNs, it is a technique that could be used for content 
distribution.  Michalakis et al., present a system for ensuring content integrity for untrusted 
peer-to-peer CDNs~\cite{michalakis2007ensuring}.  This system, Repeat and 
Compare, use attestation records and a number of peers act as verifiers.  More recently, Levy et al., 
introduced Stickler, which is a system that allows content publishers to guarantee the end-to-end 
authenticity of their content to users~\cite{levy2015stickler}.  Stickler includes content publishers 
signing their content, and users verifying the signature without having to modify the browser.  Unfortunately, 
systems like Stickler do not protect against an adversary that wishes to learn information about content, clients, 
or publishers; \system{} is complementary to Stickler.\\
%There has been prior work in securing CDNs against DDoS attacks; Gilad 
%et al. introduce a DDoS defense called CDN-on-Demand~\cite{gilad2016cdn}.  In this work they 
%provide a complement to CDNs, as some smaller organizations cannot afford the use of CDNs and 
%therefore do not receive the DDoS protections provided by them.  CDN-on-Demand is a software 
%defense that relies on managing flexible cloud resources as opposed to using a CDN provider's 
%service.\\
\vspace{0pt}
\textbf{Security Issues in CDNs.} More prevalent in the literature than defense are attacks on CDNs.  Liang et al., studied 
20 CDN providers and found that there are many problems with HTTPS practice in CDNs~\cite{liang2014https}.  Some of these 
problems include: invalid certificates, private key sharing, neglected revocation of stale certificates, and 
insecure back-end communications; the authors point out that some of these problems are fundamental issues due to 
the man-in-the-middle characteristic of CDNs.  Similarly, Zolfaghari and Houmansadr found problems with HTTPS usage by 
CDNBrowsing, a system that relies on CDNs for censorship circumvention~\cite{zolfaghari2016practical}.  They found that HTTPS 
leaks the identity of the content being accessed, which defeats the purpose of a censorship circumvention tool. Jung et al., show 
that some CDNs might not actually provide much defense against flash events~\cite{jung2002flash}.  Recently, researchers have studied the privacy implications of peer-assisted CDNs; peer-assisted CDNs allow clients to cache and distribute 
content on behalf of a website.  It is starting to be supported by CDNs, such as Akamai, but the design of the paradigm
makes clients susceptible to privacy attacks; one client can infur the cross site browsing patterns of another client~\cite{jia2016anonymity}.

%Research has also covered other attacks on CDNs; Jung et al., show 
%that some CDNs might not actually provide much defense against flash events~\cite{jung2002flash}. Su and Kuzmanovic show that some CDNs are more susceptible to intential service 
%degradation, despite being known for being resilient to network outages and denial of service attacks~\cite{su2008thinning}. 
%Recently, researchers have studied the privacy implications of peer-assisted CDNs; peer-assisted CDNs allow clients to cache and distribute 
%content on behalf of a website.  It is starting to be supported by CDNs, such as Akamai, but the design of the paradigm
%makes clients susceptible to privacy attacks; one client can infur the cross site browsing patterns of another client~\cite{jia2016anonymity}.

%Additionally, researchers implemented an attack that can affect popular CDNs, such as Akamai and Limelight; this attack 
%defeats CDNs' denial of service protection and actually utilizes the CDN to amplify the attack~\cite{triukose2009content}.  %In the 
%past year, researchers have found forwarding loop attacks that are possible in CDNs, which cause requests to be served repeatedly, which 
%subsequently consumes resources, decreases availability, and could potentially lead to a denial of service attack~\cite{chen2016forwarding}.
%\paragraph{Measuring and Mapping CDNs.} As CDNs have increased in popularity, and are predicted to grow even more~\cite{predict}, much research has 
%studied the deployment of CDNs.  Huang et al., have mapped the locations of servers, and evaluated the server availability for two CDNs: 
%Akamai and Limelight~\cite{huang2008measuring}.  More recently, Calder et al., mapped Google's infrastructure; this included 
%developing techniques for mapping, enumerating the IP addresses of servers, and identifying associations between clients and clusters of 
%servers~\cite{calder2013mapping}. Scott et al., develop a clustering technique to identify the IP footprints of CDN deployments; this analysis
% also analyzes network-level interference to aid in the identification of CDN deployments~\cite{scott2016satellite}.  In 2017, researchers 
%conducted an empirical study of CDN deployment in China; they found that it is significantly different than in other parts of the world 
%due to their unique economic, technical, and regulatory factors~\cite{xue2017cdns}. 
%Other measurement studies on CDNs have focused on characterizing and quantifying flash crowds on CDNs~\cite{wendell2011going}, inferring 
%and using network measurements performed by a large CDN to identify quality Internet paths~\cite{su2009drafting}, and measuring object size distributions and 
%request characteristics to optimize caching policies~\cite{berger2017adaptsize}.
