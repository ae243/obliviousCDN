\section{Limitations}
\label{sec:discussion}

%We discuss the limitations
%of \system{} and possible avenues for future work. 

%\textbf{Limitations.} 
\system{} inhibits some CDN performance optimizations.  For
example,
%many CDNs perform HTTPS re-writes on cached content, but this can only be 
%done if the CDN has access to plaintext content.  Similarly, 
the CDN needs the 
decrypted content to perform minimizations on HTML, CSS, and Javascript files.  While 
this increases performance in traditional CDNs, content caching around the world is the greatest benefit to 
performance, which \system{} preserves.

%\textbf{CDNs operated by content hosts.} 
\system{}
assumes that the entities operating the proxies and delivering content are
distinct from the publisher. In many cases, however---particularly
for large content providers such as Netflix, Facebook, and Google---the
content provider operates their own CDN; in these cases, \system{} will
not be able to obfuscate the content from the CDN operator, since the content host
and the CDN are the same party.  %Similarly, because the CDN operator is the same
%entity as the original server, it also knows the keys that are shared with the clients.
%As a result, the CDN could also discover the keys and identify both
%the content, as well as which clients are requesting which content.

%\paragraph{Exit proxies run by volunteers.} In the description of \system{} in Sections 
%\ref{sec:design} and \ref{sec:protocol}, we assume we are running the exit proxies, but 
%the design of the system also allows for clients to run exit proxies.  Exit proxies are not 
%trusted with client identities and information, which allows for volunteers to run exit proxies.  
%The addition of an exit proxy follows the protocol in consistent hashing for when a new node 
%joins; some keys would be transferred to the new exit proxy, and clients' mapping of 
%exit proxies will be updated.  This allows for the load to be split among more proxies and 
%increases the geographic diversity of the exit proxies.
%\paragraph{Better mixing with proxy selection} In the current implementation of
%\system{}, a client's requests are all directed through the same \system{} proxy.
%Instead, a client could use a proxy autoconfiguration (PAC) file, which could direct
%requests through different proxies depending on characteristics such as which URL
%is being requested. The selection of proxies might even be randomized.  Randomizing
%the proxy that serves a particular client request ensures that no single proxy knows
%all of the requests that any particular client is making; additionally, it may make
%it more difficult for a CDN to identify the group of requests coming from a single
%client, since a client's requests would mixed among multiple proxies. 
%\textbf{Legal questions and political pushback.} Recent cases surrounding
%the Stored Communications Act in the United States raise some questions over
%whether a system like \system{} might face legal challenges from law
%enforcement agencies. To protect user data against these types of challenges,
%Microsoft has already taken steps such as moving user data to data centers in
%Germany that are operated by entities outside the United States, such as
%T-Systems. It remains to be seen whether \system{} would face similar
%hurdles, but similar systems in the past have faced scrutiny and pushback from law
%enforcement.
