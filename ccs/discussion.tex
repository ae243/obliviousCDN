\section{Discussion}
\label{sec:discussion}

In this section, we discuss the various technical, political, and legal limitations
of OCD, as well as possible avenues for future work. 

\paragraph{CDNs operated by content hosts.} The design of \system{}
assumes that the entities operating the proxies and delivering content are
distinct from original content provider. In many cases, however---particularly
for large content providers such as Netflix, Facebook, and Google---the
content provider operates their own CDN cache nodes; in these cases, \system{} will
not be able to obfuscate the content from the CDN operator, since the content host
and the CDN are the same party.  Similarly, because the CDN operator is the same
entity as the original server, it also knows the keys that are shared with the clients.
As a result, the CDN cache nodes could also discover the keys and identify both
the content, as well as which clients are requesting which content.

\paragraph{Better mixing with proxy selection.} In the current implementation of
\system{}, a client's requests are all directed through the same \system{} proxy.
Instead, a client could use a proxy autoconfiguration (PAC) file, which could direct
requests through different proxies depending on characteristics such as which URL
is being requested. The selection of proxies might even be randomized.  Randomizing
the proxy that serves a particular client request ensures that no single proxy knows
all of the requests that any particular client is making; additionally, it may make
it more difficult for a CDN to identify the group of requests coming from a single
client, since a client's requests would mixed among multiple proxies. 

\paragraph{Chosen plaintext attacks.} A CDN operator could attempt to
determine whether a particular URL was being accessed by sending requests
through specific \system{} proxies and observing the corresponding obfuscated
requests and responses in the CDN cache logs. Blinding the clients' requests
with a random nonce that is added by the proxy should prevent against this
attack. We also believe that such an attack reflects a stronger attack: from a
law enforcement perspective, receiving a subpoena for {\em existing} logs and
data may present a lower legal barrier than compelling a CDN to attack a
system.

\paragraph{Defending against spoofed content updates.} Because the CDN cache
nodes do not know either the content that they are hosting or the URLs
corresponding to the content, an attacker could masquerade as an origin server
and could potentially push bogus content for a URL to a cache node. There are
a number of defenses against this possible attack. This simplest solution is
for CDN cache nodes to authenticate origin servers and only accept updates
from trusted origins; this approach is plausible, since many origin servers already
have a corresponding public key certificate through the web PKI hierarchy.  An additional
defense is to make it difficult for to discover which obfuscated URLs correspond
to which content that an attacker wishes to spoof; this is achievable by design.
A third defense would be to only accept updates for content from the same origin
server that populated the cache with the original content.

\paragraph{Legal questions and political pushback.} Recent cases surrounding
the Stored Communications Act in the United States raise some questions over
whether a system like \system{} might face legal challenges from law
enforcement agencies. To protect user data against these types of challenges,
Microsoft has already taken steps such as moving user data to data centers in
Germany that are operated by entities outside the United States, such as
T-Systems. It remains to be seen, of course, whether \system{} would face similar
hurdles, but similar systems in the past have faced scrutiny and pushback from law
enforcement.
