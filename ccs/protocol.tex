\section{\system{} Protocol}
\label{sec:protocol}

\subsection{Publishing Content}

\subsection{Retrieving Content}

\subsection{Partial Deployment}
\label{sec:partial}
\system{} should be partially deployable in the sense that if only some of the content publishers participate or only some of the CDNs participate, then 
the system should still provide protections.  We have two different partial deployment plans, and both provide protections for those 
publishers, CDNs, and clients that use \system{}. 

{\bf Plan 1.}
One option for deploying \system{} is to ensure there is some set S of content publishers the participate fully in the 
system.  These publishers obfuscate their content, identifiers, and certificates, and most importantly, only have 
obfuscated data stored on the CDNs cache nodes.  Recall that there are n shared keys, resulting in n replicas of the 
content that {\it appear} to the CDN as different content (because each replica is encrypted with a different key).  This 
allows the minimum set of publishers S to be relatively small.  We discuss the security tradeoffs with different 
values of S in Section \ref{sec:analysis}.  This partial deployment plan protects the set of content publishers S and it 
partially protects the privacy of the clients accessing the content created by the set of publishers S.  It does not 
protect the clients' privacy as completely as full participation of all publishers in \system{} because the CDN can 
still view cross site browsing patterns among the publishers that are not participating. It is important to note though, that 
because the clients are behind proxies, the CDN cannot individually identify users.  The CDN can attribute requests to proxies, but 
not to clients.  

{\bf Plan 2.} 
It is reasonable to believe that some content publishers are skeptical of \system{} and prioritize performance 
and availability.  Therefore, they should have the option to gradually move towards full participation by pushing 
both encrypted and plaintext content to the CDN.  In this partial deployment plan, we see some set of publishers 
fully participating with only encrypted content, some other set of publishers partially participating with both 
encrypted and plaintext content, and some last set of publishers that are not participating.  Unfortunately, if 
a publisher has both encrypted and plaintext content at a cache node, and some event causes a flashcrowd --- 
the CDN sees a significantly larger spike in accesses to certain content --- then the CDN can correlate the access 
spike on encrypted and plaintext content for the same publisher.  In order to prevent this deanonymization of the 
content publisher, we can utilize multiple CDNs.  The publisher can spread replicas over different CDNs such that 
the encrypted replicas are on one CDN and the plaintext replicas are on a different CDN.  In this case the publisher 
is not susceptible to flashcrowds correlations and can still partially join the system.

\subsection{Optimizations}
While there are some optimizations that CDNs typically perform today that would not be possible with \system{}, the architecture 
of \system{} allows for new optimizations that are not possible in existing CDNs.  Here we first outline some ways in which \system{} 
can be optimized in terms of performance, and then we point out what performance enhancements CDNs would not be able to do with 
\system{}.

{\bf Pre-Fetch DNS Responses.} One way to increase the performance of \system{} is to pre-fetch DNS responses at 
the proxies.  This would allow the proxy to serve each client request faster because it would not have to send 
as many DNS requests.  Pre-fetching DNS responses would not take up a large amount of space, but it also 
would not be a complete set of all DNS responses.  Additionally, if the content is moved between cache nodes 
at the CDN, then DNS response must also change; therefore, the pre-fetched DNS responses should have a 
lifetime that is shorter than the lifetime of the content on a cache node.

{\bf Load Balance Proxy Selection.} As the proxy performs a number of operations on the client's behalf, it 
runs into the possibility of being overloaded.  With \system{}, a client can be redirected to different 
proxies based on load; this can be implemented with a PAC file, which allows 
a client to access different proxies for different domains.  In addition to being a performance benefit, 
this could also prevent a country from blocking the set of proxies that all of the country's citizens use; if 
this occurs, then the citizens can be redirected to a different proxy.   

On the other hand, CDNs become more limited in some of their actions when following \system{}'s design.  For example, 
many CDNs perform HTTPS re-writes on content that they cache, but this can only be done if the CDN has access to the 
decrypted content.  Similarly, the CDN needs the decrypted content to perform minimizations on HTML, CSS, and Javascript 
files.  Any algorithms used internally to distribute content to certain caches based on what the content is can no longer 
be used in \system{} because the CDN does not know what the content is. 
