\section{Background}
\label{sec:background}
Before describing how \system{} works, we outline how a CDN typically operates, as well as some of 
the policy issues that can arise when data is stored in multiple jurisdictions.

\subsection{CDN}
Content Delivery Networks (CDNs) provide content caching as a service to content publishers.  A 
content publisher may wish to use a CDN provider for a number of reasons:

\begin{itemize}
\item CDNs cache content in geographically distributed locations, which allows for localized 
data centers and faster download speeds.
\item CDNs use caches, which reduces the load on the content publisher's server.
\item CDNs typically provide usage analytics, which can help a content publisher get a better 
understanding of usage.
\item CDNs provide a high capacity infrastructure, and therefore provide higher availability, 
lower network latency, and lower packet loss.  
\item CDNs' data centers have high bandwidth, which allows them to handle and mitigate DDoS attacks better 
than the content publisher's server.
\end{itemize}

CDN providers usually have a large number of edge servers on which content is cached; for example, 
Akamai has more than 216,000 servers in over 120 countries around the world~\cite{akamai_facts}.  
Having more edge servers in more locations increases the probability that a cache is geographically 
close to a client, and could reduce the end-to-end latency.  Having content closer to the client can also 
reduce the likelihood of some kinds of attacks, such as BGP (Border Gateway Protocol) hijacking.  When a client requests a web page, 
the CDN serves it from its cache, one of its edge servers.  If the requested page's content is 
not in its cache, then the request is forwarded to the content publisher's server, the CDN 
caches the response, and returns the content to the client. 

Because the CDN distributes content across many servers, it needs a way to direct the client 
to the edge server that contains the correct content.  There are two different techniques commonly used 
today to do so: 1) DNS redirection and 2) anycasting. DNS redirection occurs when the authoritative DNS 
server receives the DNS request from the client (after locally resolving the domain), and redirects the client 
to the correct edge server by responding with the IP address of the edge server that holds the correct content. 
Figure \ref{fig:dns_redirection} shows an example of DNS redirection. Certain 
optimizations can be made in selecting an edge server in the case that there are multiple edge servers 
with cached copies of the content; these optimizations can be made based on availability, location, and 
network conditions~\cite{krishnamurthy2001use}.  Whereas DNS redirection relies on the client's local DNS 
resolver, and makes assumptions about where a client is from the location of the local DNS resolver, 
anycast does not.  When anycast is used, the CDN's DNS resolver returns a set of IP addresses that 
are on the most-preferred BGP (Border Gateway Protocol) path between the client's ISP and the CDN's 
edge servers that announce the prefix associated with the requested domain.

\begin{figure}[t]
\centering
\includegraphics[width=.5\textwidth]{akamai_dns_example}
\caption{Example of DNS redirection in Akamai~\cite{su2009drafting}.  (1) Client queries local DNS resolver 
for \url{images.pcworld.com}, (2) Local DNS attempts name translation by contacting the name server 
of \url{pcworld.com}, and it returns a CNAME entry \url{images.pcworld.com.edgesuite.net}, where \url{edgesuite.net} 
is a domain owned by Akamai.  Two more DNS redirections occur, first for the \url{akam.net} domain and then to 
a1694.g.akamai.net, where 1694 is PCWorld's customer number, (3) the Local DNS resolver contacts \url{akamai.net} 
name server, (4) which forwards the Local DNS resolver to one the high-level Akamai DNS servers, and (5) then subsequently 
forwards to one of the low-level Akamai DNS servers, which returns the IP address of edge servers close to the client.  (6) 
the client then contacts these edge servers for the content. }
\label{fig:dns_redirection}
\end{figure}

The CDN tries to serve content directly from its edge servers (unless the content is not currently 
cached).  As a result, if a client requests a web page over HTTPS, the CDN terminates the TLS 
connection.  Despite this practice, CDN providers are taking measures to increase the security of 
their networks.  Akamai has established the Akamai Secure Content Delivery Network, which runs separately 
from the rest of Akamai's systems~\cite{akamai_ssl}.  It was developed to comply with the Payment Card Industry Data
Security Standards, and while it does comply with these standards, Akamai still essentially acts as a 
Man in the Middle and intercepts (and terminates) the SSL session.  More recently, Cloudflare has 
introduced Keyless SSL, which allows TLS keys to be stored on machines administered by the content 
publishers~\cite{cloudflare_keyless}.  Unfortunately, Cloudflare will still be able to access the decrypted 
content before it is sent to the client.

\subsection{Legal Precedent and Cross-Border Data Storage}
\label{sec:legal}
We have seen in recent news that the United States government has issued secret 
data requests to CDNs, particularly Cloudflare~\cite{cloudflare_nsl}.  An ongoing case, CREDO \& Cloudflare v. 
Jefferson Sessions is being fought as a follow-up to the years-long National Security Letters that 
Cloudflare has received.  

While government access of data at a CDN could compromise a client's privacy, it becomes a more complex issue when the data being cached is geographically 
distributed. This is clearly illustrated in the following example.  There is a content publisher in 
country X, and she's a customer of a CDN, so her content is replicated at cache nodes in many 
different countries.  The CDN is headquartered 
in country Y and operates cache nodes around the world.  In this scenario it is not clear which government can ask the CDN for information; for 
example, a government adversary may wish to learn the identity of the owner of the content, or which clients are accessing 
this content.  Country X could demand the information of the CDN by arguing that the content is originating 
from their country; Country Y could argue for the access to the data by stating that the CDN falls under their 
law.  Lastly, another country may request the information because the content is replicated and stored within 
their country.  The fact that CDNs distribute content and store it around the world opens the possibility of 
many governments demanding access to publisher and client information.

The stakeholders in this 
example are the content publisher, the CDN, and the Internet users --- and each of these entities differ in what 
they have at stake.  Alice may be punished for publishing controversial content (such as content that 
goes against the current regime); the CDN 
may be held liable for controversial information (or copyright infringing content); the Internet users' 
privacy could be leaked.  Each stakeholder should be interested (and possibly worried) about the 
consequences of overreaching government access.  \system{} is a novel design that allows technologists to play 
a role in the way data is governed, and to protect users, operators, and publishers from an overreaching government (or 
conflicting jurisdictional policies).
