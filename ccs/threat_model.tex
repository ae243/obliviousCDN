\section{Threat Model and Security Goals}
\label{sec:threat}
In this section, we describe our threat model, outline the capabilities of the 
attacker, and introduce the goals and protections provided by \system{}.

\subsection{Threat Model}
\label{sec:attacker}
Our threat model involves a passive, but powerful adversary.  This adversary is 
within the CDN, which includes two different types of attackers: 1) the CDN operator 
itself and 2) a government (or similar).  

In the case of the CDN operator being the adversary, he might try to make inferences.  
He could be an inside attacker or an insider that is compelled to provide data. 

In the case of an adversarial government or nation-state, the attacker could compel 
the CDN to divulge information, such as access logs or content.  This is a realized 
attacker, as we know that this has actually already occured, and which was discussed 
in Section \ref{sec:legal}~\cite{cloudflare_nsl}.

We address an attacker who wants to learn what content each client is accessing; this 
could mean learning either the identifier of the content, such as a URL, or the actual 
content of the web page.  Additionally, we are concerned with a passive attacker with 
the goal of surveillance of compromising the privacy of content publishers or Internet 
users.  An active attacker that attempts to modify and/or delete data is out of the 
scope of this work.  % AE: do we have to explicitly say we assume availability from the CDN?

\subsection{Security and Privacy Goals for \system{}}
\label{sec:goals}
In addition to protecting against an attacker such as that described in Section 
\ref{sec:attacker}, \system{} should provide protections for the different stakeholders.  
Each stakeholder, in this case the content publisher, the CDN, and the client, each have 
different risks, and therefore should have different protections.  Here we outline 
each stakeholder's risks and protections.

{\bf Origin Server.} The content publisher may want to publish sensitive or controversial 
content.  For example, perhaps he wants to publish information that goes against the current 
regime in his country.  An adversary could trace the content cached by a CDN back to the 
publisher, and then that publisher could subsequently be punished.  \system{} provides a 
degree of publishing anonymity; a CDN operator or overreaching government cannot determine 
the publisher based on information at the CDN.

{\bf CDN.} The CDN provider may be worried about the content it is caching because there 
have been recent debates on who is liable for the content.  CDNs may be at risk for being held 
liable for content that they don't produce, and that they may not be aware they are distributing.  
\system{} provides deniability to a CDN.  In the presence of a warrant or a subpoena, the CDN 
cannot technically provide any information about whether they are distributing certain content.  An
example is copyrighted content --- the CDN would not know they are caching copyrighted content and 
subsequently couldn't be held liable for it.

{\bf Client.} Clients, or end-users, are currently risking their privacy; CDNs can see their 
browsing patterns and which web pages they are visiting.  They are vulnerable to an insider at 
the CDN from snooping on internal data, as well as to a government adversary that demands access 
to the CDN's data.  \system{} provides privacy protections by hiding which client is accessing 
which content at the CDN.  In addition, it hides cross site browsing patterns, which a CDN 
is unique in having access to.  Some CDNs block legitimate Tor users because they are 
trying to protect cached content from attacks; for example, Akamai blocks Tor users~\cite{khattak2016you}.    \system{} would prevent 
privacy-concious Tor users from being blocked at CDNs.  Lastly, some CDNs, due to their ability 
to view cross site browsing patterns, could de-anonymize Tor users~\cite{cloudflare_tor}, but \system{} would 
prevent a CDN from compromising the anonymity of Tor users.

A strength of \system{} is that it protects the origin server, the CDN itself, and the client, whereas 
existing systems, such as Tor, only protect the client.

% where do we include this information? Proxy: 1) Protects clients by blinding, etc., 2) Jurisdictional protections by only being vulnerable to a single country’s subpoena (and this puts a smaller set of clients at risk than all CDN customers)
