\section{Threat Model and Security Goals}
\label{sec:threat}
In this section, we describe our threat model, outline the capabilities of the 
attacker, and introduce the design goals and protections provided by \system{}.

\subsection{Threat Model}
\label{sec:attacker}
Our threat model addresses a passive, but powerful attacker.  This passive attacker can 
be described by two different models: 1) the CDN operator itself and 2) a government (or similar). 
Both types of adversaries wish to take advantage of the knowledge that a CDN has.  

We address an attacker who wants to learn what content each client is accessing; this 
could mean learning either the identifier of the content, such as a URL, or the actual 
content of the web page.  Additionally, we are concerned with a passive attacker who 
wishes to learn information that compromises the privacy of content publishers and/or Internet 
users.  An active attacker that attempts to modify and/or delete data is out of the 
scope of this work.

In the case of the CDN provider being the adversary, he can view access logs and 
plaintext content. But in the case that he cannot view the content identifiers, he can try to make inferences. He 
can infer the popularity of content based on the number of accesses and infer the web page 
from the popularity. Addiitonally, this attacker may be able to infer a web page based on the length of the 
content.  This adversary could be an inside attacker or an insider who is 
compelled to provide data. 

In the case of an adversarial government or nation-state, the attacker could compel 
the CDN to divulge information, such as access logs or content.  This adversary can 
serve an overreaching subpoena or National Security Letter.  This is a realized 
attacker, as we know that this has actually already occured, and which was discussed 
in Section \ref{sec:background}~\cite{cloudflare_nsl}.

\subsection{Security and Privacy Goals for \system{}}
\label{sec:goals}
To protect against the attackers described in Section 
\ref{sec:attacker}, we highlight the design goals for \system{}. 
Each stakeholder, in this case the content publisher, the CDN, and the client, each have 
different risks, and therefore should have different protections.  All three stakeholders 
can be protected by preventing CDNs from learning information, decoupling content distribution from trust, and 
maintaining the performance benefits of a CDN while reducing the probability of attacks.  Our design goals are listed in Table \ref{tab:design_goals}, 
and we further discuss our design decisions in Section \ref{sec:design}.

\begin{table*}[t!]
\centering
\begin{tabular}{| l | l |} 
 \hline
 {\bf Design Goal} & {\bf Design Decision} \\ 
 \hline\hline
 Prevent CDN from knowing information & (1) encrypt content   \\ 
  (Section \ref{sec:obfuscate_content})                          & (2) obfuscate URL \\ \hline
 Closed system of proxies & (1) decouple content distribution from decision of trust via proxies  \\ 
  (Section \ref{sec:proxies})  & \\ \hline
 Key use and management & (1) $n$ shared keys, for $1 < n < |proxies|$ \\ 
   (Section \ref{sec:keys})                     & (2) secrecy in URL obfuscation \\ 
 \hline
\end{tabular}
\caption{Design goals and the corresponding design choices made in \system{}.}
\label{tab:design_goals}
\end{table*}

\paragraph{Prevent the CDN from knowing information.} First, and foremost, the CDN 
should not have access to all the information that was outlined 
in Section \ref{sec:background}.  By limiting the information that the CDN knows, \system{} limits 
the amount of information that an adversary can learn or request.  \system{} should hide 
content, content identifiers, and remove links between clients and their content requests, as well 
as remove links between content (and content identifiers) and the content publisher.  If the CDN 
does not know what content it is caching, who is requesting it, or who has created it, then an 
inside attacker will not be able to learn valuable information, and the CDN will not be able to supply 
a government adversary with the requested data.

\paragraph{Closed system of proxies.} There have been many legal battles over which government is allowed access 
to which data; for example, data can be stored in Country X, but belonging to an organization in Country Y, and 
the data is about a person in Country Z.  It is unclear which of these countries can legally demand 
the data with a subpoena or warrant.  The issue becomes much more complex when the specific laws 
and policies of the different countries are conflicting.  Perhaps Country X has much stronger data privacy 
guarantees and enforcement than Country Y or Z.  A recent approach taken by Microsoft was to establish 
a datacenter in Germany, which is technically under the control of the Deutsche Telekom subsidiary 
T-Systems~\cite{microsoft_germany}.  This was deployed in hopes of preventing the United States government from serving Microsoft with a subpoena 
for data stored in Germany, where German citizens (or others) can request to have their data stored.  Unfortunately, this issue has 
been debated in courts with varying outcomes; in a current legal battle, Google 
has been ordered to comply with the warrant, despite the data requested, emails, are stored abroad~\cite{google_warrant}.
  To complement these legal battles, \system{} should take 
these conflicting jurisdictional issues into account.  Additionally, the system should be able to protect 
the privacy of clients' locations; while addressing data privacy concerns in the client's jurisdiction, \system{} 
should not reveal more information about clients.  It also provides the ability for clients 
to hide their fine-grained location, while still following the policies of the jurisdictions in which 
they reside.

\paragraph{Key use and management.} \system{} should be able to achieve the previously mentioned security 
and privacy goals while not introducing new attacks.  More specifically, \system{} should maintain the 
caching benefits of a traditional CDN while still reducing 
the probability of attacks occuring, and reducing the probability of any information leakage in 
the case of an attack.

%{\bf Origin Server.} The content publisher may want to publish sensitive or controversial 
%content.  For example, perhaps he wants to publish information that goes against the current 
%regime in his country.  An adversary could trace the content cached by a CDN back to the 
%publisher, and then that publisher could subsequently be punished.  \system{} provides a 
%degree of publishing anonymity; a CDN operator or overreaching government cannot determine 
%the publisher based on information at the CDN.

%{\bf CDN.}  CDNs may be at risk for being held 
%liable for content that they don't produce, and that they may not be aware they are distributing.  
%\system{} provides deniability to a CDN.  In the presence of a warrant or a subpoena, the CDN 
%cannot technically provide any information about whether they are distributing certain content.  An
%example is copyrighted content --- the CDN would not know they are caching copyrighted content and 
%subsequently couldn't be held liable for it.

%{\bf Client.} CDNs can see their 
%browsing patterns and which web pages they are visiting.  They are vulnerable to an insider at 
%the CDN from snooping on internal data, as well as to a government adversary that demands access 
%to the CDN's data.  \system{} provides privacy protections by hiding which client is accessing 
%which content at the CDN.  In addition, it hides cross site browsing patterns, which a CDN 
%is unique in having access to.  Some CDNs block legitimate Tor users because they are 
%trying to protect cached content from attacks; for example, Akamai blocks Tor users~\cite{khattak2016you}.    \system{} would prevent 
%privacy-concious Tor users from being blocked at CDNs.  Lastly, some CDNs, due to their ability 
%to view cross site browsing patterns, could de-anonymize Tor users~\cite{cloudflare_tor}, but \system{} would 
%prevent a CDN from compromising the anonymity of Tor users.

A strength of \system{} is that it protects the origin server, the CDN itself, and the client, whereas 
existing systems, such as Tor, only protect the client.

% where do we include this information? Proxy: 1) Protects clients by blinding, etc., 2) Jurisdictional protections by only being vulnerable to a single country’s subpoena (and this puts a smaller set of clients at risk than all CDN customers)
