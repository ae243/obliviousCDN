\section{Introduction}
\label{sec:intro}

While government access of data at a CDN could compromise a client's privacy, it becomes a more complex issue when the data being cached is geographically 
distributed. This is clearly illustrated in the following example.  There is a content publisher in 
country X, and she's a customer of a CDN, so her content is replicated at cache nodes in many 
different countries.  The CDN is headquartered 
in country Y and operates cache nodes around the world.  In this scenario it is not clear which government can ask the CDN for information; for 
example, a government adversary may wish to learn the identity of the owner of the content, or which clients are accessing 
this content.  Country X could demand the information of the CDN by arguing that the content is originating 
from their country; Country Y could argue for the access to the data by stating that the CDN falls under their 
law.  Lastly, another country may request the information because the content is replicated and stored within 
their country.  The fact that CDNs distribute content and store it around the world opens the possibility of 
many governments demanding access to publisher and client information.

The stakeholders in this 
example are the content publisher, the CDN, and the Internet users --- and each of these entities differ in what 
they have at stake.  Alice may be punished for publishing controversial content (such as content that 
goes against the current regime); the CDN 
may be held liable for controversial information (or copyright infringing content); the Internet users' 
privacy could be leaked.  Each stakeholder should be interested (and possibly worried) about the 
consequences of overreaching government access.  \system{} is a novel design that allows technologists to play 
a role in the way data is governed, and to protect users, operators, and publishers from an overreaching government (or 
conflicting jurisdictional policies).
