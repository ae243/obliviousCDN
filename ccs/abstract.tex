\begin{abstract}

There has been an increase in concern over where data is located, as 
this can affect which governing authority can access it.  This is exacerbated 
by the prevalence of Content Distribution Networks (CDNs) because they can cache 
content in many different countries, regardless of who that data pertains to.  While many of 
the battles over stored data privacy have played out in the courts, technology 
can be designed to complement the legal system.  In this work, we analyze the privacy 
issues a CDN faces in the presence of an inside attacker or overreaching government, and 
design \system{}, which provides oblivious content distribution.  \system{} allows 
CDNs to provide the performance benefits of content distribution and caching, while 
hiding what the content is and who is accessing it.  We describe the protocol for publishing 
retrieving content using \system{} and show that the performance overhead is negligible.  

\end{abstract}
