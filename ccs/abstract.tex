\begin{abstract} Recent episodes of governments' attempts to access user data
have drawn increased concern to where a user's data is located. The location
of a user's data  can affect which governing authorities can access it.
Because Content Distribution Networks (CDNs) can cache content in many
different countries, often without regard to the owner of the data, a user's data
may reside in a country that differs from their own. Many  battles over stored
data privacy have played out in the courts, with legal opinions rendering
essentially ambiguous outcomes thus far. This paper offers a technical
contribution to this tussle. We first analyze the privacy  issues a CDN faces
in the presence of a government (or insider) who would wish to access
information about what data is being stored on the CDN and who is accessing
it. To mitigate the resulting threats, we design and implement \system{}, a
system provides {\em oblivious content distribution}, a property that allows
CDNs to provide the performance benefits of content distribution and caching,
while  preventing the CDN from knowing what content that is stored on the CDN
or who is accessing it. We design a the protocol for publishing retrieving
content using \system{} and, through a prototype implementation and
evaluation, show that the performance overhead is negligible.

\end{abstract}
