\section{Attacks Against \system{}}
\label{sec:sec}

We analyze and discuss how \system{} defends against different attacks.  Table 
\ref{tab:sec_table}
shows what security and privacy features \system{} provides in comparison to other related 
systems.

\begin{table}[t!]
\footnotesize
\centering
\begin{tabular}{| l | c | c | c |} 
\hline
 {} & Preserves  & Protects & Preserves \\ 
 {} & Integrity & Client & Confidentiality \\
 {} & at CDN & Identity & at CDN \\
\hline
 Stickler~\cite{levy2015stickler} & \checkmark & {} & {}\\ 
 R \& C~\cite{michalakis2007ensuring} & \checkmark & {} & {}\\
 Tor~\cite{dingledine2004tor} & {} & \checkmark & {} \\
 {\bf OCDN} & {} & {\bf \checkmark} & {\bf \checkmark} \\
\hline
\end{tabular}
\caption{The security and privacy features offered by related systems.  To our knowledge, 
\system{} is the first to address confidentiality at the CDN.}
\label{tab:sec_table}
\end{table}

\textbf{Popularity Attacks.}  An attacker who has gained access to CDN cache logs can learn information about how often 
content was requested.  Because not all content is requested uniformly, the 
attacker could potentially correlate the most commonly requested content with 
very popular webpages.  While this does not allow the CDN to learn which 
clients are accessing the content, it can reveal information about what content 
is stored on the CDN cache nodes.  \system{} handles this type of attacker by making 
the distribution of content requests appear uniform.  The content publisher (of popular 
content) generates multiple encodings of their content and URLs, and encrypts each one 
with the shared key $k$, such that they have multiple, different-looking 
copies of their content.  All of the content copies are pushed to the CDN and the key is 
shared with the exit proxy.\footnote{This also provides load balancing for exit proxies 
that hold the shared key $k$ for the popular webpage because it distributes the load
across multiple exit proxies (where each of these exit proxies are responsible for 
one of the encodings).}  %Now, the popular content does not appear as popular, 
%and it makes it difficult for an attacker to infer the popularity of the content.

\textbf{Chosen Plaintext Attacks.} An attacker can attempt to
determine whether a particular URL was being accessed by sending requests
through specific \system{} proxies and requesting access to the CDN cache logs, 
which contain the corresponding obfuscated
requests and responses. Blinding the clients' requests
with a random nonce that is added by the proxy prevents against this
attack. %We also believe that such an attack reflects a stronger attack: from a
%law enforcement perspective, receiving a subpoena for {\em existing} logs and
%data may present a lower legal barrier than compelling a CDN to attack a
%system.

\textbf{Traffic Analysis Attacks.} If a CDN itself is malicious and is attempting 
to learn information about the content and/or clients, the CDN may act as a client 
in the system.  In this attack the (malicious) client sends a request for content 
and the CDN can correlate the request with a content access at the CDN because they have 
knowledge of both the CDN logs and the requests they are making as a client.  \system{} 
defends against this by using the exit proxies as {\it natural} mixes; each exit proxy is receiving 
different requests from different clients and then forwarding the requests on to the CDN.  
These exit proxies naturally mix the requests enough that the CDN cannot conduct traffic analysis
 and determine which request corresponds to which content on the CDN's cache nodes.\footnote{For clarification, the exit proxies 
are not a type of mixnet~\cite{chaum1981untraceable}, they are simply receiving requests and then forwarding them, and due to the variety of 
requests, they provide natural mixing characteristics.}  There has 
also been numerous studies that have proposed and evaluated defenses against traffic 
analysis attacks~\cite{wright2009traffic,rackoff1993cryptographic}; \system{} could implement one of these solutions (such as slight timing delays) at the exit 
proxy.

\textbf{Spoofed Content Updates.} Because the CDN does not know either the content that it is hosting or the URLs
corresponding to the content, an attacker could masquerade as an origin server
and could potentially push bogus content for a URL to a cache node. There are
a number of defenses against this possible attack. The simplest solution is
for CDN cache nodes to authenticate origin servers and only accept updates
from trusted origins; this approach is plausible, since many origin servers already
have a corresponding public key certificate through the web PKI hierarchy.  %An additional
%defense is to make it difficult for an attacker to discover which obfuscated URLs correspond
%to which content that an attacker wishes to spoof; this is achievable by design.
%A third defense would be to only accept updates for content from the same origin
%server that populated the cache with the original content.

\textbf{Timing Attacks.} An attacker who is passively observing traffic could potentially 
correlate requests and responses based on timing information.  To address this type of attack, 
\system{} could employ techniques used in previous research, such as implementing timing 
delays, at different exit proxies.

\textbf{Sybil Attacks.} An adversary who runs {\it many} exit proxies can learn information 
about many clients and many content requests as they are responsible for encrypting/decrypting 
{\it many} requests and responses.  Previous work has analyzed the security of DHTs in this 
context~\cite{geambasu2009experiences,wolchok2010defeating}.  \system{} can employ a few different defenses to limit the probability and size 
of a Sybil attack.  To limit the number of exit proxies running on a single machine, \system{} can 
limit the number of exit proxies with the same IP address (which is a part of the exit proxy's self-certifying 
identifier) to one; therefore, the attack becomes more 
expensive as the number of machines the adversary must control increases.\footnote{Note also that this 
countermeasure prevents two or more exit proxies from being behind the same NAT.}  This defense can be 
expanded to entire network prefixes; for example, if a large (malicious) organization owns an entire prefix, 
they could launch a Sybil attack using various IP addresses within their network.  \system{} could 
limit the number of IP addresses in a given prefix to either one (which may result in a smaller set of 
exit proxies) or some small number (in which case the size of the Sybil is extremely small and cannot achieve 
its goal of being in a certain location on the hashing ring).  

\textbf{Flashcrowds.}  A flashcrowd is a large spike in traffic to a specific web
page. An attacker
could see that some content on the CDN has just seen a surge in traffic and correlate that with 
other information (for example, major world events).  This leaks information about what content the 
CDN is caching.  Fortunately, the design of \system{} can defend against this type of inference attack.  
The exit proxy can cache content in the time of a flashcrowd, such that the CDN (and therefore the attacker) 
does not see the surge in traffic.\footnote{This raises billing issues because the CDN can’t charge as much if edge servers don’t see as many requests for the origin; fortunately, RFC 2227 describes a solution for this~\cite{rfc2227}.}  
