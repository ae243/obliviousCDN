\section{Background}
\label{sec:background}

We outline how CDNs operate, including information they have access to. We also discuss some of the ongoing legal questions that CDNs currently face.

\vspace{-2mm}
\subsection{Content Delivery Networks}
CDNs cache content as a service to content publishers. A 
content publisher may wish to use a CDN provider for several reasons. 1) CDNs cache content in geographically distributed locations, which allows for localized data centers and faster download speeds; 2) CDNs typically provide usage analytics, which can help a publisher gain insight into clients; 3) CDNs provide a high capacity infrastructure, and therefore provide higher availability, lower network latency, and lower packet loss; and 4) CDNs' data centers have high bandwidth, which allows them to mitigate DDoS attacks better than a content publisher's server.

%\begin{itemize}[noitemsep]
%\item CDNs cache content in geographically distributed locations, which allows for localized data centers and faster download speeds.
%\item CDNs typically provide usage analytics, which can help a content publisher gain insight into clients.
%\item CDNs provide a high capacity infrastructure, and therefore provide higher availability, lower network latency, and lower packet loss.  
%\item CDNs' data centers have high bandwidth, which allows them to handle and mitigate DDoS attacks better than a content publisher's server.
%\end{itemize}

CDN providers usually have a large number of servers that cache content;
for example, Akamai has more than 216,000 servers in over 120 countries around
the world~ \cite{akamai_facts}.  Having more edge servers in more locations
increases the probability that a cache is geographically close to a client,
which helps reduce the end-to-end latency, as well as the likelihood of
certain kinds of attacks, such as BGP (Border Gateway Protocol) hijacking.
When a client requests a web page, the closest edge server to the client that
contains the content is identified, and the content is served from that edge
server.  Most often, this edge server is geographically closer to the client
than the content publisher's server, thus allowing the client to retrieve the
content faster. If the requested page's content is not in one of the CDN's
caches, then the request is forwarded to the content publisher's server; the
CDN caches the response and returns the content to the client.

%\begin{figure}[t]
%\centering
%\includegraphics[width=.375\textwidth]{plain_cdn_new2}
%\caption{The relationships between clients, the CDN, and content publishers in 
%CDNs today.}
%\label{fig:basic_cdn}
%\end{figure}

%\vs%pace{-4mm}
\subsection{What CDNs Can See}
\label{sec:info}
%\vspace{-2mm}

Because the CDN interacts with both content publishers and clients, it is in a unique position to learn information about the content they serve, the clients who access the content, and the content publishers that use the CDN.

\textbf{Content.}  CDNs typically have access to both the content they distribute and the identifier (\ie, URL) First, the CDN must know the URL to locate and serve the appropriate data. The CDN not only knows about the URL; it also 
has access to the plaintext content in order to perform optimizations to increase performance (\eg, minimizing CSS, HTML, and JavaScript).  
They can also inspect content to upgrade a connection to HTTPS.%; we discuss how 
%\system{} handles these types of optimizations
%in Section \ref{sec:discussion}
In addition, requesting content via HTTPS does not hide any information from the CDN, as the CDN terminates TLS connections on behalf of the publisher.  This means that not only does the CDN know the content and the URL, but it also it knows public and private keys, as well as certificates associated with the content it caches.  

\textbf{Client information.} Clients retrieve content directly from the
CDN's edge servers. These requests reveal information about the client's location and what the client is accessing. CDNs can also see each client's cross-site browsing patterns: CDNs host content for many different publishers, which allows them to see content requests for content published by different publishers.  Akamai caches enough content around the world to see up to 30\% of global Internet 
traffic~\cite{akamai_global_traffic}. The implications of a CDN having access to
this much information has led to publicly disclosed information requests~\cite{cloudflare_nsl}.
%these National Security Letters demanded information collected by the CDN and also included a gag order, which prohibits the CDN from publicly announcing the information request.  

\textbf{Publisher information.} A CDN must know information
about their customers, the content publishers. The CDN often holds the publisher's keys (including the private key), and certificates.  This has led to doubts 
about the integrity of content because a CDN can impersonate the publisher from the client's point of view~\cite{levy2015stickler}.



\subsection{Open Legal Questions}
%  In the 
%copyright realm, the European Commission is considering legislation that would remove safe harbor protection against copyright law for 
%online service providers if they host infringing content, regardless of the provenance of that content. 

Various parties are battling in the courts over cases that pertain to user
data requests and intermediary liability.  Intermediary liability would impose
criminal liability on an Internet platform (or a CDN) for the content it
provides on behalf of its customers or users. In this section, we highlight
some of these cases, which illuminate a challenge that CDNs face: by knowing
all the content that they distribute, CDNs may be burdened with the legal
responsibility for the actions of their customers and clients.

%There are numerous open questions in the legal realm regarding which government can request data stored in different countries, which 
%has led to much uncertainty.  A series of recent events have illustrated this uncertainty.  In the struggle over government access to 
%user data, cases such as {\it Microsoft vs. United States} (often known as the ``Microsoft Ireland Case'') concerns whether the United 
%States Government should have access to data about U.S. citizens stored abroad, given that Microsoft is a U.S. corporation.  
\textbf{User Data Requests.} Third parties have asked CDNs for user data. The Cloudflare CDN has been required
to share data with the FBI~\cite{cloudflare_nsl}; similarly, leaked NSA documents showed that the government agency ``collected information `by exploiting inherent 
weaknesses in Facebook's security model' through its use of the popular Akamai content
delivery network''~\cite{facebook_surv}.

\textbf{Intermediary Liability.}
Recently, questions on intermediary liability have been in the spotlight.  For example, many groups, including the Recording Industry Association of America (RIAA) and the Motion Picture Association of America (MPAA), have targeted CDNs with takedown notices for alleged copyright infringement, trademarks, and patent rights; CDNs are a more convenient target of these takedown notices than 
the content provider because oftentimes the content provider is either located in a jurisdiction where it is difficult to enforce the takedown, 
or it is difficult to determine the owner of the content \cite{medium_copyright,eff_copyright}.
Although Section 230 of the Communications Decency Act protects intermediaries,
such as CDNs, from being held
liable for the content they distribute, there have been cases where CDNs are forced
to remove content. This happened in 2015 involving the RIAA, Cloudflare, and Grooveshark \cite{techdirt_copyright}. In 2017, a district court ruled that Cloudflare is not protected from anti-piracy injunctions by the Digital Millennium Copyright Act (DMCA)~\cite{stack_copyright}.

The role of a CDN as an intermediary has also come into question in recent legislation, including a 2017 German hate speech law~\cite{netenforementact} and a U.S. law known as FOSTA-SESTA~\cite{fosta_sesta}. Both laws raise the spectre of potentially holding CDNs liable for the content that they distribute.%, despite the CDN not being a party in the content publishing.

%In October 2017, Germany passed a new law that imposes large fines, upwards of five million euros, on social media companies that do not take down illegal, racist, or slanderous comments and posts within 24 hours.  The law targets companies such as Facebook, Google, and Twitter, but could also apply to smaller companies, which could be serviced by CDNs.  In the latter case, it is an open question whether this new law also applies to CDNs.  In the United States, the SESTA bill would make Internet platforms liable for their user's illegal comments and posts.  SESTA would hold CDNs liable for the content that they distribute (despite the CDN not being a party in the content publishing); these types of laws can naturally lead to overblocking, where an intermediary errs on the side of caution and censors more content than it needs to. Such a law may also set a precedent for the censoring other types of content that are unpopular but legal. 
