\section{Threat Model and Security Goals}
\label{sec:threat}
In this section, we describe our threat model, outline the capabilities of the 
attacker, and introduce the design goals and protections provided by \system{}.

\subsection{Threat Model}
\label{sec:attacker}

Our threat model is a passive, but powerful adversary.  We are concerned with an 
adversary who has many different capabilities.  A potential adversary could gain 
access to the CDNs logs, which can contain client IP addresses and URLs.  Additionally, 
he could join \system{} as either a client or any number of clients; he is also capable 
of joining the system as an arbitrary number of exit proxies.  The adversary could 
also act as an origin server (a content publisher).  The potential attacker not only has the 
power to perform any of these actions, but can do these simultaneously.  For example, the 
adversary could join as a client, an exit proxy, and request access to the CDN's logs.  The 
goal of this type of adversary is to learn about the content being stored at the CDN and/or 
learn about which clients are accessing which content.  

This adversary has many strong capabilities making him a powerful adversary.  There has been 
precedent of this type of attacker; for example, governments have demanded access to CDNs' 
data~\cite{cloudflare_nsl}.  A possible adversary is a government requesting logs from the CDN, but 
the government could also be colluding with a CDN.  Additionally, the adversary could be a CDN operator, 
himself.  

As mentioned, \system{} addresses a {\it passive} adversary; active attackers are out of the 
scope of this work.  Prior work on securing CDNs has introduced methods to handle an actively 
malicious adversary by preserving the integrity of content stored on CDN cache nodes~\cite{levy2015stickler}.  
We do not address an adversary that tampers, modifies, or deletes any data, content, or requests.  

\subsection{Security and Privacy Goals for \system{}}
\label{sec:goals}
To protect against the attackers described in Section 
\ref{sec:attacker}, we highlight the design goals for \system{}. 
Each stakeholder, in this case the content publisher, the CDN, and the client, each have 
different risks, and therefore should have different protections.  All three stakeholders 
can be protected by preventing CDNs from learning information, decoupling content distribution from trust, and 
maintaining the performance benefits of a CDN while reducing the probability of attacks.  Our design goals are listed in Table \ref{tab:design_goals}, 
and we further discuss our design decisions in Section \ref{sec:design}.

%\begin{table*}[t!]
%\centering
%\begin{tabular}{| l | l |} 
% \hline
% {\bf Design Goal} & {\bf Design Decision} \\ 
% \hline\hline
% Prevent CDN from knowing information & (1) encrypt content   \\ 
%  (Section \ref{sec:obfuscate_content})                          & (2) obfuscate URL \\ \hline
% System of proxies & (1) decouple content distribution from decision of trust via proxies  \\ 
%  (Section \ref{sec:proxies})  & (2) hiding client identities \\ \hline
% Key use and management & (1) key churn \\ 
%   (Section \ref{sec:keys})                     & (2) secrecy in URL obfuscation \\ 
% \hline
%\end{tabular}
%\caption{Design goals and the corresponding design choices made in \system{}.}
%\label{tab:design_goals}
%\end{table*}

\paragraph{Prevent the CDN from knowing the content it is caching} First, and foremost, the CDN 
should not have access to all the information that was outlined 
in Section \ref{sec:background}.  By limiting the information that the CDN knows, \system{} limits 
the amount of information that an adversary can learn or request.  \system{} should hide 
the content as well as the URL associated with the content.  If the CDN 
does not know what content it is caching then the CDN will not be able to supply an adversary 
with the requested data and it will have a strong argument as to why it cannot be held 
liable for its customers' content.

\paragraph{Prevent the CDN from knowing the identity of users accessing content} CDNs can currently see clients' 
browsing patterns and which web pages they are visiting. \system{} should provide privacy protections by hiding which client is accessing 
which content at the CDN.  In addition, it should hide cross site browsing patterns, which a CDN 
is unique in having access to.  Some CDNs block legitimate Tor users because they are 
trying to protect cached content from attacks; for example, Akamai blocks Tor users~\cite{khattak2016you}.    \system{} would prevent 
privacy-concious Tor users from being blocked at CDNs.  Lastly, some CDNs, due to their ability 
to view cross site browsing patterns, could de-anonymize Tor users~\cite{cloudflare_tor}, but \system{} would 
prevent a CDN from compromising the anonymity of clients.

%\paragraph{Prevent an adversary from ...} \annie{I'm trying to get at protecting against an adversary that 
%joins the system as a client/exit/publisher, and requests access from the CDN}

%\paragraph{System of proxies} There have been many legal battles over which government is allowed access 
%to which data; for example, data can be stored in Country X, but belonging to an organization in Country Y, and 
%the data is about a person in Country Z.  It is unclear which of these countries can legally demand 
%the data with a subpoena or warrant.  The issue becomes much more complex when the specific laws 
%and policies of the different countries are conflicting.  Perhaps Country X has much stronger data privacy 
%guarantees and enforcement than Country Y or Z.  A recent approach taken by Microsoft was to establish 
%a datacenter in Germany, which is technically under the control of the Deutsche Telekom subsidiary 
%T-Systems~\cite{microsoft_germany}.  This was deployed in hopes of preventing the United States government from serving Microsoft with a subpoena 
%for data stored in Germany, where German citizens (or others) can request to have their data stored.  Unfortunately, this issue has 
%been debated in courts with varying outcomes; in a current legal battle, Google 
%has been ordered to comply with the warrant, despite the data requested, emails, are stored abroad~\cite{google_warrant}.
%  To complement these legal battles, \system{} should provide CDN functionality, while not having to trust 
%the CDN to keep client information private.  It should also provide the ability for clients 
%to hide their location and make their requests unlinkable to their identity.

%\paragraph{Key use and management} \system{} should be able to achieve the previously mentioned security 
%and privacy goals while not introducing new attacks.  More specifically, \system{} should maintain the 
%caching benefits of a traditional CDN while still reducing 
%the probability of attacks occuring, and reducing the probability of any information leakage in 
%the case of an attack.

%%%%%{\bf Origin Server.} The content publisher may want to publish sensitive or controversial 
%content.  For example, perhaps he wants to publish information that goes against the current 
%regime in his country.  An adversary could trace the content cached by a CDN back to the 
%publisher, and then that publisher could subsequently be punished.  \system{} provides a 
%degree of publishing anonymity; a CDN operator or overreaching government cannot determine 
%the publisher based on information at the CDN.

%{\bf CDN.}  CDNs may be at risk for being held 
%liable for content that they don't produce, and that they may not be aware they are distributing.  
%\system{} provides deniability to a CDN.  In the presence of a warrant or a subpoena, the CDN 
%cannot technically provide any information about whether they are distributing certain content.  An
%example is copyrighted content --- the CDN would not know they are caching copyrighted content and 
%subsequently couldn't be held liable for it.

%{\bf Client.} CDNs can see their 
%browsing patterns and which web pages they are visiting.  They are vulnerable to an insider at 
%the CDN from snooping on internal data, as well as to a government adversary that demands access 
%to the CDN's data.  \system{} provides privacy protections by hiding which client is accessing 
%which content at the CDN.  In addition, it hides cross site browsing patterns, which a CDN 
%is unique in having access to.  Some CDNs block legitimate Tor users because they are 
%trying to protect cached content from attacks; for example, Akamai blocks Tor users~\cite{khattak2016you}.    \system{} would prevent 
%privacy-concious Tor users from being blocked at CDNs.  Lastly, some CDNs, due to their ability 
%to view cross site browsing patterns, could de-anonymize Tor users~\cite{cloudflare_tor}, but \system{} would 
%%%%%prevent a CDN from compromising the anonymity of Tor users.

A strength of \system{} is that it protects the origin server, the CDN itself, and the client, whereas 
existing systems, such as Tor, only protect the client.

\subsection{Performance Considerations}
As one of the primary functions of a CDN is to make accessing content faster and more 
reliable, \system{} should not hinder performance significantly.  The performance of \system{} will 
be worse than that of traditional CDNs because it is performing more operations on content, but \system{} 
is offering confidentiality, whereas traditional CDNs are not. Therefore, the system can incur some small overhead, 
but not so much that it defeats the use of a CDN in the first place.  Additionally, \system{} should be able to 
scale with number of clients using the system, as well as with the growing number of web pages on the internet.
