\section{Introduction}
\label{sec:intro}

%Governments are increasingly using their authority to access data from
%their citizens and foreigners, even when this data may be stored overseas.  For
%example, in a
%recent case, the United States government tried to compel Microsoft to
%surrender data about U.S. citizens, even when the data itself was stored
%abroad~\cite{microsoft_ireland}. Users may also face the converse problem, where access to their data
%may depend on the laws of the country where their data is stored. Recent work,
%for example, highlights the possibility that governments may move data across
%borders to facilitate surveillance~\cite{arnbak2015loopholes}.  

As Content Distrubtion Networks (CDNs) host an increasing amount of content
from a diversity of publishers, they are fast becoming targets of requests for
data about their content and who is requesting it, as well as requests for
takedown of material ranging from alleged copyright violations to offensive
content. The shifting legal and political landscape suggests that CDNs may
soon face liability for the content that they host. For example, the European
Union has been considering laws that would remove safe harbor protection on
copyright infringement for online service providers if they do not deploy
tools that can automatically inspect and remove infringing content~\cite{eu-copyright}.  
In the United States, various laws under consideration threaten
aspects of Section 230 of the Communications Decency Act, which protects CDNs
from federal criminal liability for the content that they host. Tussles
surrounding speech, from copyright violations to hate speech, are currently
being addressed in the courts, yet the legal outcomes remain ambiguous and
uncertain, sometimes with courts issuing opposing rulings in different cases.
Regardless of these outcomes, however, CDNs are increasingly in need of {\em technical}
protections against the liability they might face as a result of content that they
(perhaps unwittingly) serve.

Towards this end, we design and implement a system that allows clients to
retrieve web objects from one or more CDNs, while preventing the CDNs from
learning either (1)~the content that is stored on the cache nodes; or (2)~the
content that clients request. We call this system an {\em oblivious
CDN}~(\system{}), because the CDN is oblivious to both the content it is
storing and the content that clients request.

\system{} allows clients to request individual objects with identifiers that
are encrypted with a key that is shared by an open proxy and the origin server
that is pushing content to cache nodes, but is not known to any of the CDN
cache nodes.  To do so, the origin server publishes content obfuscated with a
shared key, which is subsequently shared with a proxy that is responsible for
routing requests for objects corresponding to that URL.  A client forwards a
request for content through a set of peers (\ie, other OCDN clients) in a way
that prevents both other clients  and the CDN from learning the client
identity or requested content.  After traversing one or more client
proxies, an exit proxy transforms the URL that it receives from a client to an
obfuscated identifier using the key that is shared with the origin server
corresponding to the identifier.  Upon receiving that request from the exit
proxy, the CDN returns the object corresponding to the object identifier; that
object is encrypted with a key that is shared between the origin and the
proxy. This approach allows a user to retrieve content from a CDN without
any node in the CDN ever seeing the URL or the corresponding content, or even knowing
the identity of the client that made the original request. Using \system{} requires
only minimal modification to existing clients; clients can also configure aspects
of the system to trade off performance for privacy.

Ensuring that the CDN operator never learns information about either (1)~what
content is being stored on its cache nodes or (2)~which objects individual
clients are requesting is challenging, due to the many possible inference
attacks that a CDN might be able to mount. For example, previous work has
shown that even when web content is encrypted, the retrieval of a collection
of objects of various sizes can yield information about the web page that was
being retrieved~\cite{panchenko2016website, cai2012touching}. Similarly, URLs
can often be inferred from relative popularity in a distribution of web
requests, even when the requests themselves are encrypted. Additionally, the
\system{} design assumes a strong attack model (Section~\ref{sec:threat}),
whereby an adversary can request logs from the CDN, interact with \system{} as
a client, a proxy, or a publisher, and mount coordinated attacks that depend on
multiple such capabilities. Our threat model does not include active attempts
to disrupt the system (\eg, blocking access to parts of the system, mounting
denial of service attacks), but it includes essentially any type of attack
that involves observing traffic and even directly interacting with the system
as a client or a publisher.

The design of \system{} (Section~\ref{sec:design}) under such a strong attack
model entails many unique aspects and features. Because the system allows any
client to join as a proxy, even setting up the infrastructure is challenging.
For example, an attacker could try to join the system as a proxy with the
intent of proxying for specific web content, in an attempt to either disrupt
or surveil those requests. To counter this threat, \system{} uses consistent
hashing to map object identifiers (\ie, URLs) to the proxy responsible for
ultimately routing traffic to the CDN that hosts the object; to ensure that
publishers only communicate keys to the proxies responsible for their content,
each proxy must prove its identity to the respective publisher using a proof
that relies on a self-certifying identifier. 

Requesting and retrieving content, a process that we describe in detail in
Section~\ref{sec:protocol}, is challenging since neither the CDN nor the proxy
must know which client originated a request for a specific piece of content.
The key exchange between an origin server and its respective proxy protects
the confidentiality of both the content and the identifier (\ie, the URL) from
the CDN. To obfuscate the source of the original request, clients construct a
source route to an {\em exit proxy}, but the route can be prepended with
proxies that precede the client who originated the request. To defend against
various inference attacks, as well as to balance load, the \system{} design
allows publishers to use multiple CDNs to distribute the same content,
ensuring that no single CDN has access to information such as the relative
popularity distribution of all objects. To ensure that no single proxy learns
the request pattern for a single object, as well as to balance load, the
design also can also use consistent hashing to assign a set of proxes to a
single object. 

The design of \system{} against a strong adversary is a major
contribution of this work; additionally, we have also implemented \system{}
(Section~\ref{sec:implementation}) and publicly released the source code.
Section~\ref{sec:performance} studies the performance implications of the
tradeoffs between performance and privacy, as well as how \system{} performs
relative to a conventional CDN; Section~\ref{sec:sec} analyzes how \system{}
defends against threats from our adversary.
\ref{sec:discussion} describes various
limitations and possible avenues for future work, Section~\ref{sec:related}
discusses related work, and Section~\ref{sec:conclusion} concludes.


\if 0
While government access of data at a CDN could compromise a client's privacy,
it becomes a more complex issue when the data being cached is geographically
distributed. This is clearly illustrated in the following example.  There is a
content publisher in  country X, and she's a customer of a CDN, so her content
is replicated at cache nodes in many  different countries.  The CDN is
headquartered  in country Y and operates cache nodes around the world.  In
this scenario it is not clear which government can ask the CDN for
information; for  example, a government adversary may wish to learn the
identity of the owner of the content, or which clients are accessing  this
content.  Country X could demand the information of the CDN by arguing that
the content is originating  from their country; Country Y could argue for the
access to the data by stating that the CDN falls under their  law.  Lastly,
another country may request the information because the content is replicated
and stored within  their country.  The fact that CDNs distribute content and
store it around the world opens the possibility of  many governments demanding
access to publisher and client information.

The stakeholders in this example are the content publisher, the CDN, and the
Internet users --- and each of these entities differ in what  they have at
stake.  Alice may be punished for publishing controversial content (such as
content that  goes against the current regime); the CDN  may be held liable
for controversial information (or copyright infringing content); the Internet
users'  privacy could be leaked.  Each stakeholder should be interested (and
possibly worried) about the  consequences of overreaching government access.
\system{} is a novel design that allows technologists to play  a role in the
way data is governed, and to protect users, operators, and publishers from an
overreaching government (or  conflicting jurisdictional policies).
\fi 
