\section{Introduction}
\label{sec:intro}

Governments are increasingly using their authority to access data from
their citizens and foreigners, even when this data may be stored overseas.  For
example, in a
recent case, the United States government tried to compel Microsoft to
surrender data about U.S. citizens, even when the data itself was stored
abroad~\cite{microsoft_ireland}. Users may also face the converse problem, where access to their data
may depend on the laws of the country where their data is stored. Recent work,
for example, highlights the possibility that governments may move data across
borders to facilitate surveillance~\cite{arnbak2015loopholes}.  

The rise of content distribution networks (CDNs) makes the threat to citizens'
privacy more widespread, because data may be replicated in geographically
diverse regions, and users may not have purview over where their data is
stored, or where their traffic goes. Another recent development is that the
CDNs themselves may become liable for the content that they are hosting. For
example, the European Union has been considering laws that would remove safe
harbor protection on copyright infringement for online service providers if
they do not deploy tools that can automatically inspect and remove infringing
content~\cite{eu-copyright}. Both of these tussles are currently being
addressed in the courts, yet the legal outcomes remain ambiguous and
uncertain, sometimes with courts issuing opposing rulings in different cases.
In this paper, we explore whether redesigning the CDN itself could help
resolve this tussle.


In this paper, we design and implement a content distribution network that
allows clients to retrieve web objects, while preventing the CDN cache nodes
from learning either the content that is stored on the cache nodes or the
content that clients are requesting. We call this system an {\em oblivious
CDN}~(\system{}), because the CDN itself is oblivious to both the content it is
storing and the content that clients are requesting. \system{} allows clients to
request individual objects with identifiers that are encrypted with a key that
is shared by a client proxy and the origin server that is pushing content to
cache nodes, but is not known to any of the CDN cache nodes.  To do so, the
origin server publishes multiple replicas of each object, each encrypted under
a different shared key that is subsequently shared with a corresponding client
proxy. To retrieve content, a client proxy transforms the URL that it receives
from a client to an obfuscated identifier using the key shared with the origin
server. The CDN cache node then returns the object corresponding to the object
identifier; that object is also encrypted with a key that is shared between
the origin and the proxy. This approach allows a user to retrieve content from
a CDN without the cache node ever seeing the URL or the corresponding content.
Users can use \system{} with minimal modification to their existing configuration:
merely directing a web browser to an \system{} proxy allows a client to use the
system.

Although the basic mechanisms for obfuscating the URL and the corresponding
content are relatively straightforward, ensuring that the CDN operator never
learns information about either (1)~what content is being stored on its cache
nodes or (2)~which objects individual clients are requesting is more
challenging, due to the many possible inference attacks that a CDN might be
able to run. For example, previous work has shown that even when web content
is encrypted, the retrieval of a collection of objects of various sizes can
yield information about the web page that was being fetched~\cite{panchenko2016website,
cai2012touching}. Similarly, URLs
can often be inferred from relative popularity in a distribution of web
requests, even when the requests themselves are encrypted. \system{} addresses
these challenges by creating copies of the same object that are encrypted
under different keys, and deploying multiple versions of the same object to
the same CDN cache node, with each copy being encrypted with a different key.
While this approach reduces the threat of various types of inference and also
limits the number of \system{} proxies that share a key, each CDN cache node must
store multiple copies of the same object, reducing both storage efficiency and
cache hit rates. Our evaluation explores the implications of this tradeoff, as
well as how \system{} performs relative to a conventional CDN.

We make the following contributions. First, we explore the problems that
arise when data is stored across jurisdictional boundaries and highlight the
need for a CDN that is oblivious to the content it is hosting and serving.
Second, we design and implement \system{}, a CDN that can be oblivious to the
content that it is hosting, as well as the content that clients are
retrieving. Finally, we evaluate the performance of \system{} for individual
object retrieval and cache hit rates, showing that \system{} incurs a negligible performance
overhead relative to conventional CDNs.

The rest of the paper is organized as follows. Section~\ref{sec:background} describes
the typical operation of a CDN, the types of information that CDN operators
typically know today, and the types of information we aim to obfuscate.  Section~
\ref{sec:threat} describes the threat model and security objectives for \system{}; based
on these threats, Section~\ref{sec:design} outlines various design decisions. Section~
\ref{sec:protocol} describes the protocol for both publishing and retrieving content,
and Section~\ref{sec:performance} evaluates the performance overhead of \system{}. Section~
\ref{sec:discussion} describes various limitations and possible avenues for future
work, Section~\ref{sec:related} discusses related work, and Section~\ref{sec:conclusion}
concludes.


\if 0
While government access of data at a CDN could compromise a client's privacy,
it becomes a more complex issue when the data being cached is geographically
distributed. This is clearly illustrated in the following example.  There is a
content publisher in  country X, and she's a customer of a CDN, so her content
is replicated at cache nodes in many  different countries.  The CDN is
headquartered  in country Y and operates cache nodes around the world.  In
this scenario it is not clear which government can ask the CDN for
information; for  example, a government adversary may wish to learn the
identity of the owner of the content, or which clients are accessing  this
content.  Country X could demand the information of the CDN by arguing that
the content is originating  from their country; Country Y could argue for the
access to the data by stating that the CDN falls under their  law.  Lastly,
another country may request the information because the content is replicated
and stored within  their country.  The fact that CDNs distribute content and
store it around the world opens the possibility of  many governments demanding
access to publisher and client information.

The stakeholders in this example are the content publisher, the CDN, and the
Internet users --- and each of these entities differ in what  they have at
stake.  Alice may be punished for publishing controversial content (such as
content that  goes against the current regime); the CDN  may be held liable
for controversial information (or copyright infringing content); the Internet
users'  privacy could be leaked.  Each stakeholder should be interested (and
possibly worried) about the  consequences of overreaching government access.
\system{} is a novel design that allows technologists to play  a role in the
way data is governed, and to protect users, operators, and publishers from an
overreaching government (or  conflicting jurisdictional policies).
\fi 
