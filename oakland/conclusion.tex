\section{Conclusion}
\label{sec:conclusion}

As more content is distributed via CDNs, CDNs are increasingly becoming the
targets of data requests and liability cases.  We discuss why CDNs are
powerful in terms of the information they know  and can gather, such as a
client's cross site browsing patterns.  In response to  traditional CDNs'
capabilities, we design \system{}, which provides oblivious content
distribution.  \system{} obfuscates data such that the CDN can operate without
having knowledge of  what content they are caching.  This system not only
provides protections to CDNs, but also preserves client privacy by ensuring
that the CDN and the origin server never learn the identity of clients that
make requests for content. \system{} is robust against a strong adversary who
has access to request logs at the CDN and can also join the system as a
client, publisher, or CDN. Our evaluation shows that \system{} imposes some performance
overhead due to the cryptographic operations that allow it to obliviously cache
content, but that this overhead is acceptable, particularly for the sizes of files
that make up common web workloads.
