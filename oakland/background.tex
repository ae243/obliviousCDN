\section{Background}
\label{sec:background}

We now outline how a CDN typically operates, what information it
has access to by virtue of running a CDN, and some of the ongoing legal battles
surrounding CDNs.

\subsection{Content Distribution Networks}
CDNs provide content caching as a service to content publishers.  A 
content publisher may wish to use a CDN provider for a number of reasons:

\begin{itemize}
\item CDNs cache content in geographically distributed locations, which allows for localized 
data centers, faster download speeds, and reduces the load on the content publisher's server.
\item CDNs typically provide usage analytics, which can help a content publisher get a better 
understanding of usage as compared to the publisher's understanding without a CDN.
\item CDNs provide a high capacity infrastructure, and therefore provide higher availability, 
lower network latency, and lower packet loss.  
\item CDNs' data centers have high bandwidth, which allows them to handle and mitigate DDoS attacks better 
than the content publisher's server.
\end{itemize}

CDN providers usually have a large number of edge servers on which content is cached; for example, 
Akamai has more than 216,000 servers in over 120 countries around the world~\cite{akamai_facts}.  
Having more edge servers in more locations increases the probability that a cache is geographically 
close to a client, and could reduce the end-to-end latency, as well as the likelihood of some kinds of 
attacks, such as BGP (Border Gateway Protocol) hijacking.  This is evident when a client requests a web page; the closest 
edge server to the client that contains the content is identified and the content is served from that 
edge server.  Most often, this edge server is geographically closer to the client than the content publisher's 
server, thus increasing the speed in which the client receives the content. If the requested page's content is 
not in one of the CDN's caches, then the request is forwarded to the content publisher's server, the CDN 
caches the response, and returns the content to the client. 

\begin{figure}[t]
\centering
\includegraphics[width=.5\textwidth]{plain_cdn_new2}
\caption{The relationships between clients, the CDN, and content publishers in 
CDNs today.}
\label{fig:basic_cdn}
\end{figure}

\subsection{Information Visible to CDNs}
\label{sec:info}
Because the CDN interacts with both content publishers and clients, as shown in Figure \ref{fig:basic_cdn}, it is in a unique position 
to learn an enormous amount of information.  CDN providers know information about all clients who
access data stored at the CDN, information about all content publishers that cache content at 
CDN edge servers, and information about the content itself.

\paragraph{Knowing the content}  CDNs, by nature, have access to all content that they distribute, as well as 
the URL.  First, the CDN must use the URL, which is not 
encrypted or hidden, to locate the content. Therefore, it is evident that the CDN already knows what content is 
stored in its caches.  And because CDNs provide analytics to content publishers, they keep track of cache hit 
rates, and how often content is accessed.  But the CDN does not just know about the content identifier, it also 
has access to the plaintext content.  The CDN performs optimizations on the content to increase performance; 
for example, CDNs minimize CSS, HTML, and JavaScript files, which reduces file sizes by about 20\%.  They can 
also inspect content to conduct HTTPS re-writes; we discuss how \system{} handles these types of optimizations later 
in Section \ref{sec:discussion}. In addition, requesting content via HTTPS does not hide any information 
from the CDN; if a client requests a web page over HTTPS, the CDN terminates the TLS connection on behalf of the 
content publisher.  This means that not only does the CDN know the content, the content identifier, but also knows 
public and private keys, as well as certificates associated with the content it caches.  

%Recently, the fact that CDNs 
%know the content they are distributing has made its way into the legal system.  A court order was given to Cloudflare 
%that required the CDN to search out and block publishers who use a variation of a trademark held by a group of 
%music labels~\cite{eff_cloudflare_trademark}.  Originally, the music labels went after the trademark infringing 
%website, but later the order was extended to Cloudflare; the order ``required CloudFlare to block all of its customers 
%from using domain names that contained `grooveshark,' regardless of whether those domains contained First 
%Amendment-protected speech, or had any connection with the `New Grooveshark' defendants who were the 
%targets of the actual lawsuit.'' CDNs may also run the risk of running afoul of
%copyright law: for example, recent developments in the European Union propose to
%remove safe harbor protections for some CDNs if they do not employ automated detection
%techniques for removing copyrighted content~\cite{eu-copyright}.

\paragraph{Knowing client information} Clients fetch content directly from the CDN's edge servers, which reveals 
information about the client's location and what the client is accessing.  Unique to CDNs is the fact that 
they can see each client's cross site browsing patterns.  CDNs host content for many different publishers, which allows 
them to see content requests for content published by different publishers.  This gives an enormous amount of 
knowledge to CDNs; for example, Akamai caches enough content around the world to see up to 30\% of global Internet 
traffic~\cite{akamai_global_traffic}.  And we have seen the implications of a CDN knowing this much information when Cloudflare 
went public with the National Security Letters they had received~\cite{cloudflare_nsl}. These National Security Letters 
demand information collected by the CDN and also include a gag order, which prohibits the CDN from publicly announcing 
the information request.  

\paragraph{Knowing content publisher information} A CDN must know information
about their customers, the content
publishers; the CDN keeps track of who the content publisher is and 
what the publisher's content is.  The combination of the CDN seeing all content in plaintext and the content's 
linkability with the publisher, gives the CDN even more power.  Additionally, as mentioned previously, the CDN often 
holds the publisher's keys (including the private key!), and the publisher's certificates.  This has led to doubts 
about the integrity of content because a CDN can impersonate the publisher from the client's point of view~\cite{levy2015stickler}.

\subsection{Open Legal Questions}
There are numerous open questions in the legal realm regarding which government can request data stored in different countries, which 
has led to much uncertainty.  A series of recent events have illustrated this uncertainty.  In the struggle over government access to 
user data, cases such as {\it Microsoft vs. United States} (often known as the ``Microsoft Ireland Case'') concerns whether the United 
States Government should have access to data about U.S. citizens stored abroad, given that Microsoft is a U.S. corporation.  In the 
copyright realm, the European Commision is considering legislation that would remove safe harbor protection against copyright law for 
online service providers if they host infringing content, regardless of the provenance of that content.  The Cloudflare CDN has been required
to share data with FBI \cite{cloudflare_nsl}; similarly, leaked NSA documents showed that the government agency ``collected information `by exploiting inherent 
weaknesses in Facebook's security model' through its use of the popular Akamai content delivery network'' \cite{facebook_surv}.

More recently, questions on intermediary liability have been in the spotlight.  For example, many groups, including the Recording Industry 
Association of America (RIAA) and the Motion Picture Association of America (MPAA), have started targeting CDNs with takedown notices for 
content that allegedly infringes on copyright, trademarks, and patent rights; CDNs are a more convenient target of these takedown notices than 
the content provider because oftentimes the content provider is either located in a jurisdiction where it is difficult to enforce the takedown, 
or it is difficult to determine the owner of the content \cite{medium_copyright,eff_copyright}.  While U.S. law Section 230 protects intermediaries, such as CDNs, from being held 
liable for the content they distribute, there have been cases where CDNs are forced to remove content.  This happened in 2015, as mentioned in Section \ref{sec:info}, which 
involved the RIAA, Cloudlfare, and Grooveshark \cite{techdirt_copyright}.  
And again in 2017, a district court ruled that Cloudflare is not protected from anti-piracy injuctions by the Digital Millennium Copyright Act (DMCA); the 
RIAA obtained a permanent injunction against a site known as MP3Skull, which contained pirated content, and was distributed by Cloudflare.  The ruling 
did not specify that Cloudflare was enjoined with MP3Skull under the DMCA, but rather that Cloudflare was helping MP3Skull in evading the injunction (under 
Rule 65 of the Federal Rules of Civil Procedure) \cite{stack_copyright}.

The role of a CDN as an intermediary has also come into question in the announcement of new legislation, including a new German hate speech law and 
a bill proposed by the U.S. Senate called Stop Enabling Sex Traffickers Act (SESTA).  In October 2017, Germany passed a new law that imposes large fines, upwards of five million euros,
on social media companies that do not take down illegal, racist, or slanderous comments and posts within 24 hours \cite{nytimes_hatespeech}.  The law 
targets companies like Facebook, Google, and Twitter, but could also apply to smaller companies, which could be serviced by CDNs.  In the latter case, it is 
an open question whether this new law also applies to CDNs.  In the United States, a new bill, SESTA, would make Internet platforms liable for their user's illegal comments 
and posts \cite{medium_sesta}.  SESTA would CDNs liable for the content that they distribute (despite the CDN not being a party in the content publishing); this could potentially lead to 
censoring content too much (an intermediary may be more willing to err on the side of caution--censorship).  This type of law could set a dangerous precedent 
for the censoring other types of content that are unpopular, but still legal. 

All these cases highlight a key problem that CDNs face by knowing all the content
that they distribute: it may burden them with the legal responsibility for the actions of their customers 
and clients.
