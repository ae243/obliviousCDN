\section{Security Analysis}
\label{sec:sec}
\annie{Fill me in with a security analysis/evaluation.}

{\bf Popularity Attacks.}

{\bf Traffic Correlation Attacks.}

{\bf Chosen Plaintext Attacks.} A CDN operator could attempt to
determine whether a particular URL was being accessed by sending requests
through specific \system{} proxies and observing the corresponding obfuscated
requests and responses in the CDN cache logs. Blinding the clients' requests
with a random nonce that is added by the proxy should prevent against this
attack. We also believe that such an attack reflects a stronger attack: from a
law enforcement perspective, receiving a subpoena for {\em existing} logs and
data may present a lower legal barrier than compelling a CDN to attack a
system.

{\bf Denial of Service Attacks.}

{\bf Spoofed Content Updates.} Because the CDN cache
nodes do not know either the content that they are hosting or the URLs
corresponding to the content, an attacker could masquerade as an origin server
and could potentially push bogus content for a URL to a cache node. There are
a number of defenses against this possible attack. This simplest solution is
for CDN cache nodes to authenticate origin servers and only accept updates
from trusted origins; this approach is plausible, since many origin servers already
have a corresponding public key certificate through the web PKI hierarchy.  An additional
defense is to make it difficult for to discover which obfuscated URLs correspond
to which content that an attacker wishes to spoof; this is achievable by design.
A third defense would be to only accept updates for content from the same origin
server that populated the cache with the original content.

{\bf Flashcrowds.}

{\bf Censorship.}
