\section{Design}
\label{sec:design}

We have designed Oblivious DNS (ODNS), which decouples any DNS query from the IP address 
that initiated the query, while allowing the existing DNS infrastructure to remain 
unchanged.  We first introduce an authoritative subdomain for ODNS queries. This might 
be a top-level DNS name (e.g., .odns), though for an initial prototype any authoritative 
domain will do. Our initial prototype will use .odns.princeton.edu.  The basic design 
sketch is as follows.

\begin{enumerate}
\item When the client generates a request for \url{www.foo.com}, its stub resolver generates a session key $k$, encrypts the requested domain, and appends the TLD domain \url{.odns}, resulting in $\{$\url{www.foo.com}$\}_k$\url{.odns}. 
\item The client forwards this request, with the session key encrypted under the \url{.odns} authoritative server’s public key ($\{k\}_{PK}$) in the “Additional Information” record of the DNS query to the recursive resolver, which then forwards it to the authoritative name server for \url{.odns}.  
\item The authoritative server for ODNS queries decrypts the session key with its private key and subsequently decrypts the requested domain with the session key.  
\item The authoritative server forwards a recursive DNS request to the appropriate name server for the original domain, which then returns the answer to the ODNS server.
\item The ODNS server can thus return the answer to the client’s stub resolver directly, possibly over a confidential channel such as D-TLS.
\end{enumerate}

Other name servers see incoming DNS requests, but these only see the IP address of the ODNS recursive resolver, which effectively proxies the DNS request for the original client.
