\section{Data at Rest}
\label{sec:rest}

Data at rest refers to stored data. This type of data has been described as a subset of communications; Bruce Schneier described stored data as: 

\say{...a stored message is a way for someone to communicate with himself through time~\cite{schneier2007applied}.}

While this is true in an abstract sense, the measures needed to protect the privacy of both types of data are fairly different.  An example of data at rest is a database of credit card information associated with some website.  This information is stored on some machine and is accessible to incoming requests.

Here we discuss some of technical questions regarding the privacy of data at rest, some ongoing court cases, and how (and why) systems must protect data at rest differently than data in motion.

\subsection{Questions/Concerns}  
Just as with data in motion, encryption cannot fully solve the privacy problem associated with data at rest.  If stored data is encrypted, then that data appears to be protected, but in reality the keys used for encrypting the data are often stored on the same computer or the same network as the data.  In this case, if the attacker has access to the encrypted data, then he also has access to the encryption keys, which defeats the entire point of encryption.  

There is one other difference that is worth pointing out.  A nation state adversary not only has the capability of wiretapping or snooping on data, but can also serve an organization, ISP, or company with a subpoena for access to their data.  In the case of an overreaching government legally asking for data, this makes the problem significantly harder.  This is evident in the following examples: 1) a subpeona is served to an ISP, which as discussed earlier can man-in-the-middle TLS connections, and because the ISP {\it can} see the plaintext data in motion, then it must hand over the requested to data to the nation state, 2) a subpoena is served to a CDN, which may encrypt the data at rest, but also stores the encryption keys, and {\it can} see the data and which clients are accessing it, so it must provide the requested data to the nation state.  

It is clear that technologists need to develop systems and techniques that provide protections beyond encryption, especially in the face of a government adversary, who can unknowlingly gather the data or simply demand the data.

\subsection{Legal Precedent and Questions}
For the purposes of putting data at rest laws into context, we will be focusing on how the United States governs stored data.  The Stored Communications Act was passed to prohibit a provider of an electronic communication service ``from knowingly divulging the contents of any communication while in electronic storage by that service to any person other then the addressee or intended recipient''~\cite{stored_comm}.  The law also specifies the type of content that can be accessed via a subpeona, and addresses National Security Letters (NSLs).  NSLs allow the Federal Bureau of Investigation (FBI) to compel the production of subscriber information and electronic communication transactional records, and the only requirement is that the FBI certifies that the records sought ``are relevant to an authorized investigation to protect against international terrorism or clandestine intelligence activities....'' \cite{stored_comm}.  Additionally, each NSL comes with a gag order, which prevents the electronic service provider from revealing that the data was requested.  

The NSLs provide no transparency regarding what data was demanded, but fortunately, more and more companies are going public with past NSLs.  In January 2017, Twitter published two NSLs it received in 2015 and 2016 as the gag order was lifted; the NSLs required Twitter to hand over the ``name, address, length of service, and electronic communications transactional records for all services, as well as all accounts'' pertaining to two specified accounts~\cite{twitter_gag}.  Twitter isn't alone: Yahoo recently published three NSLs that were previously under gag order and Google published eight data requests that were under gag order~\cite{yahoo_gag,google_gag}. More recently, CREDO and Cloudflare have published their respective NSLs; an on-going court case (CREDO \& Cloudflare v. Jefferson Sessions) has CREDO and Cloudflare arguing that NSLs violate their freedom of speech~\cite{cloudflare_gag}.  

Many of these companies that have recieved NSLs under gag order have publicly pushed for more transparency; Google made the statement ``We have fought for the right to be transparent about our receipt of NSLs. This includes working with the government to publish statistics about NSLs we’ve received, successfully fighting NSL gag provisions in court, and leading the effort to ensure that Internet companies can be more transparent with users about the volume and scope of national security demands that we receive.'' Along similar lines, Twitter announced ``Twitter remains unsatisfied with restrictions on our right to speak more freely about national security requests we may receive. We would like a meaningful opportunity to challenge government restrictions when ‘classification’ prevents speech on issues of public importance.''  Facebook agreed publicly: ``We'll also keep working with partners in industry and civil society to push governments around the world to reform surveillance in a way that protects their citizens’ safety and security while respecting their rights and freedoms''~\cite{cloudflare_gag}.  This shows that many Internet companies want to protect their customers' data from government access, and further highlights the need for new systems to be developed that can prevent government overreach.

Microsoft has taken a first step in protecting their customers' privacy from an overreaching government.  In 2015, Microsoft established a data center in Germany, which is shepherded by Deutsche Telekom (Germany's largest telecommunications provider), and Microsoft will only be given access to the data if either the customer or Deutsche Telekom's T-Systems subsidiary give permission~\cite{microsoft_germany}.  This is a first attempt at preventing U.S. authorities access to the data, and a leading example of how we can use and develop technology in other ways for this same purpose.

Unfortunately, not all cases are on the side of privacy; in April 2017, New York's highest court ruled that Facebook is not allowed to ask an appellate court to reject a search warrant ordering them to hand over information from hundreds of accounts~\cite{facebook_ny}.  Facebook's argument was that the search warrants were so broad that they were essentially an unconstitutional search.  This blow to Facebook's attempt at expanding privacy protections for their users motivates the general need for systems that preserve the privacy of data at rest. Technology can prevent overreaching governments from accessing such broad amounts of data, and we hope that future systems build in privacy protections for this reason. 

%[TODO: paragraph on historical case outcomes involving data at rest, reference Microsoft-Ireland Case~\cite{microsoft_ireland} and more recent ruling in Facebook v. New York \cite{facebook_ny}, and discuss what is currently being done, reference Microsoft-Germany data center~\cite{microsoft_germany}]

\subsection{Technical Approaches}
\label{sec:tech_rest}
While the academic community has seen and started to address the problem of protecting privacy in regards to data in motion, there has been significantly less work done on preserving the privacy of data at rest from a nation state.  Encrypting stored content cannot completely solve the problem, as a government can order a company to provide certain data, and currently most companies have both the encrypted content and the encryption key.  Systems can be designed such that the system owner cannot provide any data (even if ordered) because he does not know both the encryption key and the encrypted content.  We call for new techniques to prevent surveillance by a government adversary in future work.

%To our knowledge, the only work that has attempted to protect stored data from either a snooping nation state or a nation state which has served a subpoena is ObliviousCDN.  ObliviousCDN is a CDN which is designed to cache content (just as a typical CDN does), but the CDN has know knowledge of the content it is distributing~\cite{edmundson2017cdn}.  It does this by employing proxies between the client and the CDN and ensuring the CDN cache nodes only hold encrypted (or hashed) content.  Unlike the systems discussed in the previous subsection, ObliviousCDN addresses both types of data; it also provides privacy protections for data in motion.  By encrypting content before the data is transferred, and by employing a proxy between the cache node and a set of clients, it provides a level of privacy protection.  


