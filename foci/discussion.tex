\section{Discussion}
\label{sec:discussion}

Despite the amount of publicity and the push for privacy protection against government adversaries, there hasn't been as much work done on avoiding nation state surveillance.  In this section, we go through prior work and discuss how the research is relevant to avoiding surveillance, as well as where it fits in the landscape of data protection.  Table \ref{tab:current_systems} shows which proposed systems/techniques address which type of data.

\begin{table}[h]
\begin{center}
    \begin{tabular}{| l | l | l |}
    \hline
    System & Data in Motion & Data at Rest \\ \hline \hline
    Alibi Routing~\cite{levin2015alibi} & \checkmark & {} \\ \hline
    RAN~\cite{edmundson2017ran} & \checkmark & {} \\ \hline
%    ARROW~\cite{peter2015one} & \checkmark & {} \\ \hline
    SCION~\cite{zhang2011scion} & \checkmark & {} \\ \hline
    Tor~\cite{dingledine2004tor} & \checkmark & {} \\ \hline
    VPNGate~\cite{nobori2014vpn} & \checkmark & {} \\ \hline
    ObliviousCDN~\cite{edmundson2017cdn} & \checkmark & \checkmark \\ 
    \hline
    \end{tabular}
\caption{Proposed systems and techniques, and which type of data they aim to protect from nation state surveillance.}
\label{tab:current_systems}
\end{center}
\end{table}

\subsection{Data in Motion}
There has been some work on developing techniques, systems, and clean-slate Internet architectures to protect data in motion.  

Levin \ea, introduced Alibi Routing, which provides ``proofs of avoidance,'' which serves as evidence that a user's packets did not traverse a user-specified geographic region~\cite{levin2015alibi}.  This mechanism could be used to avoid (potentially with proof) a region or nation state known for surveillance.  Unfortunately, this mechanism is not very successful when trying to avoid a region that is geographically close to the client or server, and the client does not know until after the communication has occured whether or not the region was avoided.  Nevertheless, it was an initial work in country and region avoidance.  More recently, a system called RAN was introduced, which allows Internet users to avoid a user-specified country by using an overlay network~\cite{edmundson2017ran}.  This system uses traceroute measurements to determine the correct overlay network relay to use in order to avoid a country; unfortunately, these measurements cannot measure the reverse path from a server to a client, and thus provide no guarantee that the unfavorable country is not on the reverse path.

There are a number of other censorship circumvention systems that use different techniques, one example is the Tor network.  Tor is a censorship circumvention system that employs a series of relays and layered encryption~\cite{dingledine2004tor}.  Tor does protect data in motion from surveillance, but is still susceptible to correlation attacks and fingerprinting attacks~\cite{sun2015raptor,shmatikov2006timing,feamster2004location,greschbach2016effect}.  Virtual Private Networks (VPNs) have also been used to circumvent censorship; a proposed system that uses VPNs is VPNGate~\cite{nobori2014vpn}.  While VPNs protect the data in motion on a portion of the path, it leaves the data susceptible to snooping on the other portion of the path.  Additionally, use of a VPN requires the client to place trust in the VPN provider, which a nation state could demand data from.  and other censorship circumvention systems.    

SCION is a clean-slate Internet architecture that is designed to provide route control, failure isolation, and explicit trust information for end-to-end communications~\cite{zhang2011scion}.  While SCION would help preserve the privacy of data in motion, it is not deployable because it requires fundamental changes to the Internet architecture.

As we can see, there have been a number of proposed systems to avoid nation state adversaries from conducting surveillance on data in motion.  Unfortunately, we have not achieved complete privacy protections for this type of data, and we call for more research to protect citizens' data from adversarial governments.  

\subsection{Data at Rest}
While the academic community has seen and started to address the problem of protecting privacy in regards to data in motion, there has been significantly less work done on preserving the privacy of data at rest from a nation state.  

To our knowledge, the only work that has attempted to protect stored data from either a snooping nation state or a nation state which has served a subpoena is ObliviousCDN.  ObliviousCDN is a CDN which is designed to cache content (just as a typical CDN does), but the CDN has know knowledge of the content it is distributing~\cite{edmundson2017cdn}.  It does this by employing proxies between the client and the CDN and ensuring the CDN cache nodes only hold encrypted (or hashed) content.  Unlike the systems discussed in the previous subsection, ObliviousCDN addresses both types of data; it also provides privacy protections for data in motion.  By encrypting content before the data is transferred, and by employing a proxy between the cache node and a set of clients, it provides a level of privacy protection.  

There is much more work to be done on protecting data at rest from nation state surveillance.  Encrypting stored content cannot completely solve the problem, as a government can order a company to provide certain data, and currently most companies have both the encrypted content and the encryption key.  Systems can be designed such that the system owner cannot provide any data (even if ordered) because he does not know both the encryption key and the encrypted content.  We call for new techniques to prevent surveillance by a government adversary in future work.
