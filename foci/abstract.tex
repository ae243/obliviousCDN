\begin{abstract} The Internet's growth expands across international borders,
coupled with the growth of geo-replicated cloud-backed applications, has
created situations where a user's data may be stored in many countries around
the world. Typically, the location of a user's data has little correlation
with the citizenship of that user, raising questions about the laws and
policies that govern that user's data. Questions over access to user data may
meet dissonance when the laws governing a citizen in one country are at odds
with the laws governing a country where that user's data is stored.  For
example, the Safe Harbor framework was developed to allow trans-Atlantic data
flow while ensuring  the privacy of European Union citizens' data. This legal
framework, and others, such as the Stored Communications Act in the United
States, have created flashpoints for legal battles over privacy in the courts.
In this paper, we argue that {\em technology} may be able to shape privacy
debates concerning cross-border data flows and data storage, by helping to
protect the privacy of citizens' data regardless of where the client, data, or
online service resides. In this discussion, we highlight several historical
court decisions and ongoing legal tussles in this area. We then suggest that some
of the ongoing work in the security and privacy community may ultimately shape some
of these debates by offering protections for both data in motion and data at rest.
\end{abstract}