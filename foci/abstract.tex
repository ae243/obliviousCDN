\subsection*{Abstract}

As the Internet continues to grow, it does so without national borders, 
raising the question of how data should be governed, by whom, and 
under what circumstances.  Different jurisdictions follow different policies 
and regulations, particularly on the topic of data privacy.  For example, the Safe
Harbor framework was developed to allow trans-Atlantic data flow while ensuring 
the privacy of European Union citizens' data.  Along with determining the invalidity 
of the framework, the courts are where most battles on data 
privacy have been fought while technology development with jurisdictional 
differences in mind has been left to the side. In this work, 
we focus on the privacy implications of cross-border data flows and data 
storage, and how technology can be designed to protect citizens' data, regardless 
of the jurisdiction in which the client, data, or service resides.  We highlight 
some of the historical court decisions, ongoing jurisdictional debates, and evaluate 
some of the proposed systems and techniques in terms of how well they provide data protection 
for either data at rest or data in motion.  Lastly, we explain and discuss how 
technology can play a role in designing systems to facilitate data privacy in 
the face of an overreaching government or conflicting jurisdictional policies.
