\section{Introduction}
\label{sec:intro}

As the structure of the Internet continues to evolve, the ways in which 
data and communications are governed also continue to change.  Data can 
be generated by a citizen of one country, stored in a different country 
on servers belonging to a company headquartered in yet another country.  
Additionally, data can flow from one country to another by transiting a 
variety of other countries.  As the Internet is a global system spanning 
many nation states, this raises the question of how these data are 
governed, and under what circumstances are governments are allowed to access what
data.

%The policies and behavior of different nation states provide different 
%treatment of data.  These differences occur in policies regarding net 
%neutrality, copyright enforcement, data privacy protection, etc. For example, 
%some countries, such as the United States, Canada, parts of South America, parts 
%of Europe, India, Japan, and South Korea have either laws or regulations for 
%protecting net neutrality, where as other countries, such as Russia, China, and Myanmar, 
%have no net neutrality laws or regulations \cite{net_neutrality}.  Similarly, different 
%jurisdictions have different policies and enforcement of copyright infringement; both 
%the United States and China have signed onto major international copyright treaties, but 
%China has no notion of ``fair use,'' meaning someone can reproduce a copyrighted work 
%without permission if it falls under a range of excepted categories, while the United 
%States does \cite{copyright}.  We can see that as data is produced, distributed, and stored 
%around the world --- in large part due to the Internet --- we face the question of which 
%jurisdiction, and subsequently which policies, does it fall under.

A clear example of such a scenario where there are cross-jurisdictional issues is that of 
a Content Delivery Network (CDN).  A CDN consists of a set of cache nodes, which replicate 
and distribute content produced by other entities.  In our example, Alice has a blog and lives in Russia.  
She is a customer of a CDN, which replicates her 
content on a cache node in France.  The CDN is headquartered 
in the United States and operates cache nodes around the world.  There are two separate questions that should 
be addressed: 1) What type of information could an overreaching government demand? 2) Which government 
(if any) should legally be allowed to demand the data?  For the first question, some information that may be 
demanded of the CDN is the identity of a specific content owner or a list of which clients are accessing which content. 
In response to the second question, there is currently no clear answer to which jurisdiction has legal 
right to the CDN's data.  The United States could argue for the data because the CDN is headquartered 
there; Russia could argue for the data because the content publisher resides there; France could argue for the data 
because the data is located there. 

The stakeholders in this 
example are the content owner, the CDN, and the Internet users --- and each of these entities differ in what 
they have at stake.  Alice may be punished for publishing content that is politically controversial; the CDN 
may be held liable for controversial information (or copyright infringements); the Internet users' 
privacy could be leaked.  Each stakeholder should be interested (and possibly worried) about the 
consequences of overreaching government access.  In fact, we have recently seen that Cloudflare, a popular 
CDN, has been secretly required to hand over data to the FBI~\cite{cloudflare_gag}.  Similarly, leaked 
NSA documents showed that the government agency ``collected information `by exploiting inherent weaknesses 
in Facebook's security model' through its use of the popular Akamai content delivery network''~\cite{}.

In this work, we focus on the privacy implications of cross-border data flows and data 
storage, and how technology can be designed to protect citizens' data, regardless of the 
jurisdiction in which the client, data, or service resides.  In light of 
the Snowden revelations, many countries have taken 
measures to avoid United States surveillance on their citizens' Internet traffic; some of 
the ways some countries are avoiding known surveillance states include: controlling the route 
that Internet traffic takes by deploying underwater cables that avoid surveillance states~\cite{brazil}; 
requiring their citizens' data to be stored locally, so as to prevent their citizens' data from 
being stored in a country that conducts surveillance~\cite{russia}; controlling Internet paths by building 
IXPs; halting the use of technology designed and developed in surveillance states.  While some of these 
solutions are technical, many are political, and in most cases solutions come from court decisions, not 
technology.

While policy and regulation are being created and challenged, technologists have a role 
in this area too.  Technology and systems can be developed to facilitate data privacy in the 
face of an overreaching government or conflicting jurisdictional policies.  Not only can 
these systems protect Internet users' privacy, but they can also provide protections 
for content creators and content distributors (regardless of their geographic location).  

In this paper, we provide background on some of the jurisdictional differences 
on both data in motion and data at rest, discuss historical and current laws and cases of 
government access to data (in motion and at rest), as well as put these cases into the context 
of technology development. We also highlight current systems that aim to protect data privacy from nation 
state adversaries, and discuss how they stop short of circumventing nation state surveillance.  Lastly, we
 point out the gaps in this technology landscape, and 
subsequently call for future work.

These systems and techniques should be applied to two separate types of data: 1) data in 
motion and 2) data at rest.  We differentiate between these two types of data for a number of reasons.  
First, there are different policies and regulations applicable to the two types of data.  Second, 
they have different characteristics, which lead to different technological requirements to preserve 
privacy from nation-state surveillance, especially considering this surveillance can come from  
wiretapping or serving a subpoena.

The rest of the paper is organized as follows.  We discuss data in motion in more depth in Section \ref{sec:motion}, where we highlight 
historical cases, current issues, and how technology can provide protection in this domain.  We also highlight and compare
systems proposed in the literature that protect the privacy of data in motion. Section \ref{sec:rest} 
points out the important cases in regards to data at rest, as well as how technology should be designed 
differently for this type of data in comparison to data in motion.  We discuss related work in Section \ref{sec:related} and 
conclude in Section \ref{sec:conclusion}.
