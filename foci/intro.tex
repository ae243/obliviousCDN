\section{Introduction}
\label{sec:intro}

The global growth of the Internet and the proliferation of geo-replicated
services---from cloud hosting providers to content distribution networks---has
created an Internet of {\em data without borders}: A user's data can be
generated by a citizen of one country, stored in a different country on
servers belonging to a company headquartered in yet another country. When a
user stores or retrieves that data, it can flow from one country to another by
way of many other ``third-party'' countries. As a citizen's data traverses and resides
in other countries, questions naturally arise about who has access to a user's data,
and under what circumstances.

One example that presents these types of cross-jurisdictional issues is a
Content Delivery Network (CDN).  A CDN consists of a set of cache nodes, which
replicate  and distribute content produced by other entities.  Suppose that
Alice has a blog and lives in Russia.  The content of her blog may be
replicated across a CDN, which replicates her content on a cache node in France.
The company operating the CDN is headquartered in the United States and
operates cache nodes around the world. Several questions arise: (1)~Does the government
where the data resides (or where the data transits) have the ability to access that
user data?; (2)~Given that Alice is a citizen of one country, can that country gain
access to Alice's data, regardless of where the data is stored?; (3)~Is the CDN
itself bound by the laws if the country where it is headquartered, the country where
Alice is a citizen, or the country where the data is stored?
For example, a government may ask a CDN about the identity of a specific content
owner, or a list of
which clients are accessing which content.  
Currently, the legal system has produced no clear answers concerning which jurisdiction
has a right to access the user data on the CDN, and under what circumstances.  

This legal limbo creates uncertainty for content owners, CDN operators, and
users. Such legal uncertainty is not merely hypothetical: a series of recent
events have illustrated this uncertainty. In the struggle over government
access to user data, cases such as {\em Microsoft vs. United States} (often
known as the ``Microsoft Ireland Case'') concerns whether the United States
Government should have access to data about U.S. citizens stored abroad, given
that Microsoft is a U.S. corporation. In the copyright realm, the European
Commission is considering legislation that would remove safe harbor protection
against copyright law for online service providers if they host infringing
content, regardless of the provenance of that content. The Cloudflare CDN has
been required to share data with the FBI~\cite{cloudflare_gag}; similarly,
leaked  NSA documents showed that the government agency ``collected
information ``by exploiting inherent weaknesses  in Facebook's security model''
through its use of the popular Akamai content delivery network''~\cite{nsa_akamai}.

In this work, we focus on the privacy implications of cross-border data flows
and data  storage, and how technology can be designed to protect citizens'
data, regardless of the  jurisdiction in which the client, data, or service
resides.  In light of  the Snowden revelations, many countries have taken
measures to avoid United States surveillance on their citizens' Internet
traffic; some of  the ways some countries are avoiding known surveillance
states include: controlling the route  that Internet traffic takes by
deploying underwater cables that avoid surveillance states~\cite{brazil};
requiring their citizens' data to be stored locally, so as to prevent their
citizens' data from  being stored in a country that conducts
surveillance~\cite{russia}; controlling Internet paths by building  IXPs;
halting the use of technology designed and developed in surveillance states.
While some of these  solutions are technical, many are political, and in most
cases solutions come from court decisions, not  technology.

While policy and regulation are being created and challenged, technologists
have a role  in this area too.  Technology and systems can be developed to
facilitate data privacy in the  face of an overreaching government or
conflicting jurisdictional policies.  Not only can  these systems protect
Internet users' privacy, but they can also provide protections  for content
creators and content distributors (regardless of their geographic location).

In this paper, we provide background on some of the jurisdictional differences
on data privacy, discuss historical and current laws
and cases of  government access to data, as well as
put these cases into the context  of technology development. We also highlight
current systems that aim to protect data privacy from nation  state
adversaries, and discuss how they stop short of circumventing nation state
surveillance.  Lastly, we  point out the gaps in this technology landscape,
and  subsequently call for future work.

These systems and techniques should be applied to two separate types of data:
(1)~data in motion and (2)~data at rest.  We differentiate between these two
types of data for a number of reasons.   First, there are different policies
and regulations applicable to the two types of data.  Second,  they have
different characteristics, which lead to different technological requirements
to preserve  privacy from nation state surveillance, especially considering
 surveillance can be conducted via wiretapping or serving a subpoena.

The rest of the paper is organized as follows.  We discuss data in motion in
more depth in Section \ref{sec:motion}, where we highlight  historical cases,
current issues, and how technology can provide protection in this domain.  We
also highlight and compare systems proposed in the literature that protect the
privacy of data in motion. Section \ref{sec:rest} highlights the important
cases in regards to data at rest, as well as how technology should be designed
differently for this type of data in comparison to data in motion.  We discuss
related work in Section \ref{sec:related} and  conclude in Section
\ref{sec:conclusion}.
