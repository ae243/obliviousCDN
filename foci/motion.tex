\section{Data in Motion}
\label{sec:motion}

{\em Data in motion} refers to communications or data that is en route between
Internet endpoints. For example, if Alice is communicating with Bob, her
message to Bob is considered data in motion.  When Alice fetches a web page,
her request is data in motion as it travels to the web server; similarly, the
content of the web page is data in motion as it is returned to Alice. In this
section, we highlight the concerns associated with the privacy of data in
motion, discuss the applicable laws and policies governing the privacy of user
data for data in motion, and explore how specific technologies may help protect
user privacy in these settings.

\subsection{Privacy Concerns}

While encryption seems like the obvious solution to protecting the privacy of
data in motion, there are many reasons why cryptography is not a panacea.
Specifically, there are still a number of ways that a skilled adversary, such
as a nation state, can learn information.  First, the mere presence of
communication between two parties may be revealing.  Additionally, there has
been a wide variety of research in the area of website fingerprinting, which
can reveal information about the content based on size, content, and location
of third- party resources.  Some unencrypted communication may reveal
information about the content of other (encrypted) communication; DNS traffic
is extremely revealing and often unencrypted~\cite{isps_see}.  Finally, some
ISPs terminate TLS connections, thereby conducting man-in-the-middle attacks
on encrypted traffic for network management purposes~\cite{gogo}. Thus,
encryption alone cannot protect data in motion from a nation state adversary.
Additional systems are needed to conceal the nature or extent of
communications in these scenarios. These concerns are complicated by the fact
that geo-replicated services such as CDNs may also result in data traversing
many jurisdictions.  We explore these complications next and subsequently
discuss how technology may help address some of the legal ambiguity
surrounding data in motion.

\subsection{Legal Precedent and Questions}

Data in motion is subject to different data protection requirements and
privacy legislation in different jurisdictions.  For example, the European
Union (EU) has stricter data protection laws than those of the U.S.
In 1998 the, European Commission's Directive on Data Protection went
into effect, which ``would prohibit the transfer of personal data to non-
European Union countries that do not meet the European Union (EU) `adequacy'
standard for privacy protection''~\cite{safeharbor}.  The U.S. negotiated the
Safe Harbor framework, which was accepted by the Commission in 2000, and
relied on the concept of enforceable self-certifications to assure adequate
protection. This self-certification worked because the companies would
publicly commit to protecting personal data by following specific principles,
and because these commitments were public, they were enforceable by the
U.S. Federal Trade Commission.

In 2015, the highest European Court deemed the Safe Harbor framework invalid
``on the adequacy of the protection provided by the safe harbour privacy
principles and related frequently asked questions issued by the US Department
of Commerce''~\cite{safeharbor}.  The two main issues that were raised against
Safe Harbor were: (1)~excessive U.S. access to European data (largely due to
the Snowden disclosures), and (2)~no process for European citizens to address
their concerns~\cite{safeharbor_ps_diffs}.  A new framework, the Privacy
Shield, was negotiated to replace the Safe Harbor agreement and address these
concerns.  Despite the relatively recent introduction of the Privacy Shield
framework, more recent actions in the current U.S. administration may
ultimately violate it.  In January 2017, the U.S. President signed an
executive order titled ``Executive Order: Enhancing Public Safety in the
Interior of the United States,'' which mainly targeted immigration laws, but
also stated:

\begin{displayquote}
\say{Sec. 14.  Privacy Act.  Agencies shall, to the extent consistent with applicable law, ensure that their privacy policies exclude persons who are not United States citizens or lawful permanent residents from the protections of the Privacy Act regarding personally identifiable information.~\cite{exec_order}}
\end{displayquote}
\noindent
The primary takeaway from the history of cross-jurisdictional data flows is
that privacy protection laws greatly vary between jurisdictions, and that the
agreements and frameworks made are still very much in flux.  It is
still unclear what the future of EU-US data transfer will be, but
technologists have a role among these policy makers as well; technology can be
designed to protect data privacy, as well as to prevent an overreaching
government from access to a different jurisdiction's data.

While these frameworks are being negotiated between different nation states
and governing entities, they are not the only ones that are concerned about
data privacy protections.  In fact, it was an EU citizen who brought a case
against the Safe Harbor agreement, which eventually rendered it invalid; Max
Schrems, an Austrian citizen, challenged the transfer of his data to the U.S.
by Facebook~\cite{schrems}.

The Privacy Shield addresses data transfers that start in the EU and end in
the US, but the law does not mention intermediate countries on the path.  For
example, data transferred from the EU to the US may traverse Canada en route;
once traffic enters a specific country (even if neither the client nor server
are located in that country), it becomes subject to that country's policies on
surveillance and censorship.  As a result, data in motion protections can be
compromised solely because the path taken to access information traverses an
unfavorable jurisdiction.    These issues go far beyond EU personal data being
accessed by the U.S.; recent research has shown that Internet traffic suffers
``collateral damage'' simply because it is routed through a specific
jurisdiction~\cite{levis2012collateral}.  This study found that Korean traffic
to German web sites (.de domains) often suffers ``collateral damage'' in the
sense that it is censored because the path traverses an Autonomous System (AS)
known for censorship within China.  This case is generalizable to both nation
state level surveillance and censorship, and we've recently seen certain
countries, such as Brazil, take extreme measures to avoid routing their
Internet traffic through the United States for this exact
reason~\cite{brazil}.

Several ongoing debates concern the privacy of data in motion.  Some
governments are demanding backdoors in encrypted communication applications,
and other governments are trying to issue subpoenas for encrypted
communications~\cite{whatsapp_uk,signal_fbi}.  On the other side, many
companies are trying to keep their clients' data (in motion) secure; one
example of this is the increased use of end-to-end encryption in communication
applications.  End-to-end encryption is a step in the direction of protecting
privacy using technology.  It has become increasingly common for governments
to criticize technologies that use end-to-end encryption; for example, the
United Kingdom recently stated that they should have access to WhatsApp
messages when necessary (this was in response to the recent terror attacks in
London)~\cite{uk_whatsapp}.  The recent actions taken by many governments, not
just the UK, should motivate technologists to design systems with cross-
jurisdiction data flows in mind.

\subsection{Technical Approaches}
\label{sec:tech_motion}

%\begin{table}[h]
%\begin{center}
%    \begin{tabular}{| l | l | l |}
%    \hline
%    System & Data in Motion & Data at Rest \\ \hline \hline
%    Alibi Routing~\cite{levin2015alibi} & \checkmark & {} \\ \hline
%    RAN~\cite{edmundson2017ran} & \checkmark & {} \\ \hline
%    ARROW~\cite{peter2015one} & \checkmark & {} \\ \hline
%    SCION~\cite{zhang2011scion} & \checkmark & {} \\ \hline
%    Tor~\cite{dingledine2004tor} & \checkmark & {} \\ \hline
%    VPNGate~\cite{nobori2014vpn} & \checkmark & {} \\ \hline
%    ObliviousCDN~\cite{edmundson2017cdn} & \checkmark & \checkmark \\ 
%    \hline
%    \end{tabular}
%\caption{Proposed systems and techniques, and which type of data they aim to protect from nation state surveillance.}
%\label{tab:current_systems}
%\end{center}
%\end{table}

Various technologies can help protect data in motion.  One strategy is to {\em
route around} the jurisdictions that have unfavorable or ambiguous laws.
Recent work on Alibi Routing provides ``proofs of avoidance'' that as evidence
that a user's packets did not traverse a user-specified geographic
region~\cite{levin2015alibi}.   This mechanism could be used to prove that a
packet did not traverse a region or nation state known for surveillance. Yet,
this mechanism is not effective for avoiding regions that are  is
geographically close to the client or server; furthermore,  the client does
not know until after the communication has occurred whether or not the region
was avoided. It also does not provide proactive mechanisms for avoiding a
region. 

A more recent system called RAN allows  Internet users to avoid a
user-specified country using an overlay network~\cite{edmundson2017ran}.  This
system uses traceroute measurements to determine the correct overlay network
relay to use in order to avoid a country; unfortunately, these measurements
cannot measure the reverse path from a server to a client, and thus provide no
guarantee that the unfavorable country is not on the reverse path. Finally, these
systems cannot help protect user privacy in the case where the data to be protected
is hosted in the country that the client may wish to avoid; for example, many clients
simply cannot avoid the United States or certain European countries because they
are the locations where the data is hosted.

SCION is a clean-slate Internet architecture that is designed to provide route
control, failure isolation, and explicit trust information for end-to-end
communications~\cite{zhang2011scion}.  While SCION would help preserve the
privacy of data in motion, it is not deployable because it requires
fundamental changes to the Internet architecture.


Another strategy is to {\em obfuscate} the existence of communications from
the jurisdictions that may wish to discover it. One such system, Tor, uses a
series of relays and layered encryption~\cite{dingledine2004tor}.  Tor
protects data in motion from surveillance, but is still susceptible to
correlation and fingerprinting
attacks~\cite{sun2015raptor,shmatikov2006timing,feamster2004loc
ation,greschbach2016effect}.  Virtual Private Networks (VPNs) have also been
used to circumvent censorship; a proposed system that uses VPNs is
VPNGate~\cite{nobori2014vpn}.  While VPNs protect the data in motion on a
portion of the path, they leave the data susceptible to snooping on the
portion of the path that is not covered by the VPN.  Additionally, using a VPN
requires the client to trust the VPN provider, which a nation state could
demand data from.

Although there have been a number of proposed systems to avoid nation state
adversaries from conducting surveillance on data in motion.  Unfortunately, we
have not achieved complete privacy protections for this type of data. More research
is needed to address the shortcomings of these existing technologies.
